\subsection{Why object recognition is difficult}
\begin{itemize}[--]
	\item Things that make it hard to recognize objects:
	\begin{itemize}[--]
		\item \textbf{Segmentation:} Real scenes are cluttered with other objects
			\begin{itemize}[--]
				\item It's hard to tell which pieces go together as parts of the same object
				\item Parts of an object can be hidden behind other objects
			\end{itemize}

		\item \textbf{Lighting:} The intensities of the pixels are determined as much by the lighting as by the objects
		\item \textbf{Deformation:} Objects can deform in a variety of non-affine ways:
			\begin{itemize}[--]
				\item eg. a hand-written 2 can have a large loop or just a cusp
			\end{itemize}

		\item \textbf{Affordances:} Object classes are often defined by how the are used
		\item \textbf{Viewpoint:} Changes in viewpoint cause changes in images that standard learning methods cannot cope with
	\end{itemize}
\end{itemize}

\subsection{Achieving viewpoint invariance}
\begin{itemize}[--]
	\item Several approaches:
	\begin{itemize}[--]
		\item Use redundant invariant features
		\item Put a box around the object and use normalized pixels
		\item Use replciated features with pooling. This is called ``convolution neural nets''
		\item Use a hierarchy of parts that have explicit poses relative to the camera
	\end{itemize}
\end{itemize}

\subsubsection{The invariant feature approach}
\begin{itemize}[--]
	\item Extract a large, redudant set of features that are invariant under transformations
	\item With enough invariant features, there is only one way to assemble them into an object
	\item But for recognition, we must avoid forming features from parts of different objects
\end{itemize}

\subsubsection{The judicious normalization approach}
\begin{itemize}[--]
	\item Put a box around the object and use it as a coordinate frame for a set of normlized pixels
	\begin{itemize}[--]
		\item This solves the dimension-hopping problem. If we choose the box correctly, the same part of an object always occurs on the same normalized pixels
		\item The box can provide invariance to many degrees of freedom: translation, rotation, scale, shear, stretch, etc.
	\end{itemize}
	\item Chooosing the box is difficult because of: segmetnation erros, occlusion, unusual orientations
	\item We need to recognize the shape to get the box right (chicken and egg dilema)
\end{itemize}

\subsubsection{The brute force normalization approach}
\begin{itemize}[--]
	\item When training the recognizer, use well-segmented, upright images to fit the correct box
	\item At test time try all possble boxes in a range of positiosn and scales
	\begin{itemize}[--]
		\item This approach is widely used for detecting upright things like faces and house numbers in unsegmented images
		\item It is much more efficient if the recognizer can cope with some variation in position and scale so that we can use a coarse grid when trying all possible boxes
	\end{itemize}
\end{itemize}