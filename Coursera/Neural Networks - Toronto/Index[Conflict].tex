\documentclass[english, 11pt]{article}
\usepackage{../../notes}
\usepackage{lipsum}

%Global Course Variables
\newcommand{\myCourseCode}{University of Toronto}
\newcommand{\myCourseName}{Neural Networks for Machine Learning}
\newcommand{\myProf}{Geoffrey Hinton}
\newcommand{\myTerm}{Summer 2015}
\newcommand{\myLogo}{UofT-Crest-Wide.png}

%Headers
\lhead{\myCourseName}
\rhead{\fancyplain{}{\rightmark}} 

%Footers
\cfoot{\thepage}

\begin{document}
\titleHeader{\myCourseCode}{\myCourseName}{\myProf}{\myTerm}{\myLogo}

%Document information
\noindent\rule{1\columnwidth}{.5pt}
Contributors: Max Smith
\begin{center}
	Latest revision: \today
\end{center}
\toc
\abstr{Neural networks use learning algorithms that are inspired by our understanding of how the brain learns, but they are evaluated by how well they work for practical applications such as speech recognition, object recognition, image retrieval and the ability to recommend products that a user will like. As computers become more powerful, Neural Networks are gradually taking over from simpler Machine Learning methods. They are already at the heart of a new generation of speech recognition devices and they are beginning to outperform earlier systems for recognizing objects in images. The course will explain the new learning procedures that are responsible for these advances, including effective new proceduresr for learning multiple layers of non-linear features, and give you the skills and understanding required to apply these procedures in many other domains.}

%----------------------------
%Document Begins
%----------------------------

\section{Introduction}

\section{The Perceptron learning procedure}

\section{The backpropagation learning proccedure}

\section{Learning feature vectors for words}

\section{Object recognition with neural nets}
	\subsection{After calibration}
\img{.8}{sections/lec5/p.png}
\begin{itemize}
	\item Internal parameters $K$ are known
	\item $R, T$ are known - but these can only relate $C$ to the calibraiton rig.
	\item You can't estimate $P_i$ from the single image measurement of $p_i$ because it may be anywhere along the line in space.
	\item We also don't have any information in the image to tell us the scale of anything in the world.
\end{itemize}

\subsection{Recovering structure from a single view}
\begin{itemize}
	\item You can make assumptions to get relative sizes of objects in images
	\item Pick a reference plane in the scene
	\item Pick a reference direction (not parallel to the reference plane) in the scene
	\img{.8}{sections/lec5/ref.png}
	\subsubsection{Geometry}
	\img{.8}{sections/lec5/van.png}
	\item Under perspective projection, parallel lines in three-space project to convering lines in the image plane. The common point of intersection, perhaps at infinity, is called the \textbf{vanishing point}
	\begin{itemize}
		\item projection of a point at infinity (goes infinity far)
		\img{.8}{sections/lec5/vp.png}
		\item Any two parallel lines have the same vanishing point
		\item The ray from $C$ through $v$ point is parallel to the lines
		\item AN image may have more than one vanishing point
	\end{itemize}
	\item Two or more vanishing points from lines known to lie in a single 3D plane establish a \textbf{vanishing line}, which completely determines the orientation of the plane
	\img{.7}{sections/lec5/house.png}
\end{itemize}

\subsection{The Cross Ratio}
\img{.8}{sections/lec5/y.png}
\begin{itemize}
	\item $Y$ is desired height to measure
	\item Compute Y from image measurements
	\begin{itemize}
		\item You'll need more than vanishing points to compute this
	\end{itemize}
	\item \textbf{Projective Invariant}: something that does not change under projective transformations (including perspective projection)
	\img{.8}{sections/lec5/l.png}
	\item The \textbf{cross ratio} of 4 collinear points is projective invariant
	$$\frac{||P_3 - P_1||\cdot ||P_4 - P_2||}{||P_3 - P_2||\cdot ||P_4 - P_1||}$$
	\item You can permute the point ordering and it will remain true.
	\item Often called the fundamental invariant of projective geometry
	\img{.8}{sections/lec5/inf.png}
	\item When you consider the points at $\infty$, and the point where the camera would meet the ground plane $v_z$ you have enough points to use the cross ratio on our camera model.
	\item You need the same points in the world and camera system (can't mix and match)
	\item You must make some assumption to determine the relative sizes, typically assume $L$
\end{itemize}

\subsection{Horizon line}
\img{.7}{sections/lec5/hor.png}
\begin{itemize}
	\item Sets of parallel lines on the same plane lead to collinear vanishing points the line is called the \textbf{horizon}
	\img{.8}{sections/lec5/ex.png}
	\item When trying to recover the structure within the camera reference system, we can check if two linesare parallel or not
	\begin{itemize}
		\item If they do, recognize the horizon line
		\item Measure if hte 2 lines meet the horizon
		\item If they do, they are parallel in 3D
		\item Actual scale of scene is not recovered, only relative distances
	\end{itemize}
\end{itemize}

\subsection{Lines in a 2D plane}
$$ax+by+c=0; l=\begin{bmatrix}
	a\\b\\c
\end{bmatrix}$$
\subsubsection{Intersecting lines}
\img{1}{sections/lec5/intersect.png}
\begin{itemize}
	\item The intersection of lines can be calculated with: 
	$$x=l\times l'$$
\end{itemize}

\subsection{Stereo-view geometry}
\img{1}{sections/lec5/tri.png}
\begin{itemize}
	\item Two camera perspectives allow us to find position of objects through \textbf{triangulation}
	\begin{itemize}
		\item This requires a few knowns, $K_1, K_2, R, T$
	\end{itemize}
	\item Small inaccuracies with knowns can lead to situations where the lines may not intersect 
	\item Instead, we'll find where they came close enough
		$$d^2(x_1, M_1X)+d^2(x_2, M_2X)$$
		$$d:=\text{distance between two lines}$$
\end{itemize}

\subsection{Epipolar Geometry}
\img{.7}{sections/lec5/ep.png}
\begin{itemize}
	\item \textbf{Epipolar plane} (grey): intersections of baseline with image planes
	\item \textbf{Baseline} (orange): projectison of other camera center
	\item \textbf{Epipolar lines} (blue): vanishing points of camera motion direction
	\item This framepoint, allows us to consider if two pointsare related by epipolar geometry
	\item Use epipolar lines to find sharing points will greatly save computation cost as a 2D search becomes 1D
	\subsubsection{Parallel image planes}
	\item Baseline intersects the image plane at infinity
	\item Epipoles are at infinity
	\item Epipolar lines are parallel to x-axis
	\subsubsection{Forward translation}
	\img{1}{sections/lec5/f.png}
	\item When a camera moves forward the lines turn into a spiral
\end{itemize}

\section{Optimization: How to make the learning go faster}
	\subsection{Encryption as a Function}
\begin{itemize}
	\item Plaintest string $s$
	\item Encryption key $K_{enc}$
	\item Decryption key $K_{dec}$
	\item Encrypt $s$ with $K_{enc}$ to obtain ciphertext $K_{enc}(s)$
	\item Decrypt $K_{enc}(s)$ with decryption key $K_{dec}$ to reobtain $s$
	\item $K_{dec}(K_{enc}(s))=s$
	\item Encryption applies a reversible fn to some piece of data, yielding something unreadable
	\item Decryption recovers the original data from the unreadable encryption-output
	\item The encryption/decryption algorithm assumed known; the key is secret
\end{itemize}

\subsection{Substitution Ciphers}
\begin{itemize}
	\item Arbitrary 1:1 mapping of alphabet chars, using a \textbf{substitution table}
	\item All of these are vulnerable to frequency analysis:
	\begin{itemize}
		\item letter
		\item word
		\item common phrases
	\end{itemize}
	\item We could make the substitution table larger (translating $n$-gram, not chars):
	\begin{center}\begin{tabular}{l|l}
		Plaintext & Ciphertext \\
		AAA & QWE \\
		AAB & RTY \\
		AAC & ASD
	\end{tabular}\end{center}
	\item How big is substitution table?
	\begin{itemize}
		\item $A^n$ entries, where A is size of alphabet and $n$ is size of grams
	\end{itemize}
	\item Still vulnerable, but requies more text
\end{itemize}

\subsection{Substitution Rules}
\begin{itemize}
	\item Don't store table explicitly; derive table rows using substitution rule
	\begin{itemize}
		\item e..g, $s \text{XOR} k$, where $k$ is the key
		\item Remember: security level depends on size of key
		\item Key of len $b\geq 2^b$ possible keys
	\end{itemize}
	\item XOR ``flips a bit'' for input bits that correspond to key's ``1''
	\item Encrypted string should ideally show no pattern for frequency analysis attack
	\item What's the right key size?
	\begin{itemize}
		\item Who is trying to break the scheme?
		\item Is that good enough?
	\end{itemize}
\end{itemize}

\subsection{Data Encryption Standard (DES)}
\begin{itemize}
	\item DES is a block cipher with 56-bit key
	\item Data transmitted in 64-bit blocks, each may be coded independently
	\item Triple-DES, breaks up text in 3-56 bit chunks and apply DES to each with different keys
	\img{.7}{sections/lec6/des.png}
\end{itemize}

\subsection{Review of Crypto}
\begin{itemize}
	\item The key is traditional crypto is used to encode the substitution rule
	\begin{itemize}
		\item Needed to encrypt and decrypt
	\end{itemize}
	\item Key distribution is the weak link
	\begin{itemize}
		\item Hard to revoke
		\item Disastrous if ``codebook'' is compromised
		\item Hard to distribute 
		\item Impossible for the Web
	\end{itemize}
\end{itemize}

\subsection{Public-Key Cryptography}
\begin{itemize}
	\item Secure communication without key exchange
	\item Each party has a pair of related keys: \textbf{public} and \textbf{private}
	\begin{itemize}
		\item A public key is published freely
		\item A private key is shared with no one
	\end{itemize}
	\item A message enrypted witho ne can be decrypted with the other
	\item YOu can't derive one from the other (this is critical!)
	\subsubsection{Two Modes}
	\item I encrypt using the public key, and decrypt using the private key
	\img{.8}{sections/lec6/two.png}
	\item Anyone can encrypt; only $S$ can decrypt
	\item Used for data confidentiality
	\item If encrypted using private, anyone with public can decrypt
	\item Used for authenticity
	\item So if someone was able to encrypt, then they must have had private, ensuring no man-in-the-middle
\end{itemize}

\subsection{Trapdoor Functions}
\begin{itemize}
	\item Public key cryptography relies on so-called trapdoor functions
	\begin{itemize}
		\item A fn that is easy to compute, but hard to invert without special info
		\item ``easy'' and ``hard'' meant comutationally
	\end{itemize}
	\item Some poor choices: add, multiply
	\item In practice quite difficult to find good trapdoor rfunctions
	\item Most popular one is related to prime factorizaiton; others possible
	\subsubsection{Prime Factorization}
	\item $n=pq$, where $p$ and $q$ are primes
	\begin{itemize}
		\item Given $p$ and $q$, easy to compute $n$
		\item Given $n$, very hard to find $p$ and $q$
	\end{itemize}
	\subsubsection{Fermat's Little Theorem}
	\item For any prime $p$, and any integer $a$: $a^p = a(\mod p)$
	\subsubsection{Chinese Remainder Theorem}
	\item Consider $x = a_i (\mod p_i), \text{ for }i=1,\ldots, k$
	\item CRT: There's a solution for $x$ if $p_i$ are pairwise relatively prime (i.e., have no common factors greater than 1)
	\item If all $a_i$ are 1, then $x=1(\mod p_i)$
\end{itemize}

\subsection{Cryptography}
\begin{itemize}
	\item We choose two large primes: $p, q$
	\item $n=pq$
	\item Next:
	\begin{itemize}
		\item Set $\lambda = (p-1)(q-1)$
		\item Choose $e$ randomly, such that $e<\lambda$
		\item Choose $d$, such that $de=1(\mod \lambda)$
	\end{itemize}
	\item $n$ and $e$ serve as public key
	\begin{itemize}
		\item $n$ is product of primes $p, q$
	\end{itemize}
	\item $n$ and $d$ serve as private key
	\begin{itemize}
		\item Choosing $d$ requires $e$ and $\lambda$
	\end{itemize}
	\item $\text{encrypt}_{n,e}(m)=m^e (\mod n)=c$
	\item $\text{decrypt}_{n,d}(c)=c^D(\mod n)= m$
	\item Slower than DES because exponential instead of XOR or SHIFTs
	\item Will decryption always work?
	\begin{itemize}
		\item $c^d = m^{de}=m^{k\lambda + 1}$
		\item $m^{k\lambda + 1}=m(m^{(p-1)(q-1)})k$
		\item Recall from Fermat that:
		\begin{itemize}
			\item $m^{p-1}=1(\mod p)$
			\item $m^{q-1}=1(\mod q)$
		\end{itemize}
		\item Which implies:
		\begin{itemize}
			\item $m^{(p-1)(q-1)}=1(\mod p)$
			\item $m^{(p-1)(q-1)}=1(\mod q)$
		\end{itemize}
		\item Use the CRT to combine above 2 equations:
			$$m^{(p-1)(q-1)}=1(\mod n),\text{ where } n=pq$$
		\item $c^d=m(1)^k(\mod n)$
		\item $c^d=m(\mod n)$
		\item Thus, decrypted ciphertext $c=\text{msg}(m)$
	\end{itemize}
	\item Send private shared keys via expensive method, then use cheap method once key is shared. HTTPS does this.
\end{itemize}

\section{Recurrent neural networks}
	\subsection{Passing Pointers to Functions}
\begin{itemize}
	\item Pointers can be arguments to functions. For example, suppose you want a function that adds one to an integer argument passed by reference:
\begin{lstlisting}[style=C++]
void add_one(int *x){
	// MODIFIES: *x
	// EFFECTS: adds one to *x
	*x = *x + 1;
}
\end{lstlisting}
	\item If you were to call this function as so: \lstinline[style=C++]{add_one(bar);}, where \lstinline[style=C++]{bar} is a pointer to \lstinline[style=C++]{foo}
\begin{lstlisting}[style=C++]
int foo;
int *bar;
bar = &foo;

add_one(bar);
\end{lstlisting}
	\begin{itemize}
		\item The variable bar is bassed by \textbf{value}, but it's a pointer!
		\item Both bar and the copy of bar refer to the same address in memory.
	\end{itemize}
	\item You can also make the call without the ``middleman'' like: \lstinline[style=C++]{add_one(&foo);}
\end{itemize}

\subsection{Pointer question}
\begin{itemize}
	\item If you modify \lstinline[style=C++]{add_one} to:
\begin{lstlisting}[style=C++]
void add_one(int *x){
	x = x + 1;
}
\end{lstlisting}	
	\item It will increment the value \textbf{of the pointer} by one.
	\item Pointer arithmetic is done based on units of the \textbf{referent type} (the type of the objects in the list).
\end{itemize}

\subsection{Pointers vs. references}
\begin{itemize}
	\item Both allow you to pass objects by reference.
	\item Pointers require some extra syntax at calling time (\&), in the argument list (*), and with each use (*); references only require extra syntax in the argument list (\&).
	\item You can change the object to which a pointer points using arithmetic/assignment, but you cannot change the object to which a reference refers.
	\item You might wonder why you’d ever want to use pointers, since theyrequire extra typing, and allow you to shoot yourself in the foot.
	\item Why use pointers?
	\begin{itemize}
		\item Array variables are internally implemented using pointers
		\item They allow us to create structures (unlike arrays) whose size is not known in advance; we won't see that use until the last third of the course.
	\end{itemize}
\end{itemize}

\subsection{Pointers and Arrays}
\begin{itemize}
	\item Arrays are actually represented via pointers as so:
	\begin{center}
		\includegraphics{sections/lec7/array.png}
	\end{center}
	\item If you were to look at the value of the variable ``array'' (not \lstinline[style=C++]{array[0]}) you'd find that it was exactly the same as the address of \lstinline[style=C++]{array[0]}.
	\item When the argument \lstinline[style=C++]{array} is passed to the function sum, a pointer to the first element of the array is really passed and the compiler does all the work of translating something like: \lstinline[style=C++]{array[3]} into the proper arithmetic/dereference to get the right value.
\begin{lstlisting}[style=C++]
x = array[3];
// Is equivalent to:
int *tmp;
tmp = array + 3;
x = *tmp;
// Or simply:
x = *(array + 3);
\end{lstlisting}
\end{itemize}

\subsection{Indexing vs. pointer arithmetic}
\begin{itemize}
	\item Using array indexing:
\begin{lstlisting}[style=C++]
for (int i = 0; i < SIZE; ++i){
	cout << array[i] << " ";
}
\end{lstlisting}
	\item Using pointer arithmetic:
\begin{lstlisting}[style=C++]
for (int *i = array; i < array+SIZE; ++i){
	cout << *i << " ";
}
\end{lstlisting}
\end{itemize}

\subsection{Array Traversal Using Pointers}
\begin{lstlisting}[style=C++]
int strlen(char *s) {
	char *p = s;
	while (*p) ++p;
	return p - s;
}
\end{lstlisting}
\begin{itemize}
	\item \lstinline[style=C++]{*p} evalues to ``false'' if \lstinline[style=C++]{p} points to a NULL, true otherwise.
	\item \lstinline[style=C++]{++p} advances by ``one character''
	\item \lstinline[style=C++]{p-s} computes the ``number of characters'' between \lstinline[style=C++]{p} and \lstinline[style=C++]{s}
\end{itemize}

\subsection{Constants}
\begin{itemize}
	\item \lstinline[style=C++]{void strcpy(char *dest, const char *src);}
	\item \lstinline[style=C++]{const} is a \textbf{type qualifier} - something that modifies a type
	\item It means ``you cannot change this value once you have initialized it.''
	\item When you have pointers, there are two things you might change:
	\begin{itemize}
		\item The value of the pointer.
		\item The value of the object to which the pointer points.
	\end{itemize}
	\item Either (or both) can be made unchangeable:
\begin{lstlisting}[style=C++]
const T *p;			// "T" (the pointed-to object) cannot be changed
T *const p;			// "p" (the pointer) cannot be changed
const T *const p;	// neither can be changed.
\end{lstlisting}
	\item Adding \lstinline[style=C++]{const} will stop changing value mistakes, and the compiler will catch them.
	\item You can use a pointer-to-T anywhere you expect a pointer-to-const-T, but NOT vice versa
	\item That's because code that expects a pointer-to-T might try to change the T, but this is illegal for a pointer-to-const-T.
	\item However, code that expects a pointer-to-const-T will work perfectly well for a pointer-to-T; it's just guaranteed not to try to change it.
\end{itemize}

\subsection{C strings vs. C++ strings}
\begin{center}
\begin{tabular}[breaklines=true]{p{5cm}|p{5cm}|p{5cm}}
	& C string & C++ string \\
	\hline
	Library headers & 
{\begin{lstlisting}[style=C++] 
#include <string>
\end{lstlisting}} & 
{\begin{lstlisting}[style=C++]
#include string
\end{lstlisting}}\\	
	string constant &
{\begin{lstlisting}[style=C++]
constchar* hello = "hello";
\end{lstlisting}}&
{\begin{lstlisting}[style=C++]
conststring hello = "hello";
\end{lstlisting}} \\
	length &
{\begin{lstlisting}[style=C++]
strlen(hello);//5
\end{lstlisting}} &
{\begin{lstlisting}[style=C++]
hello.length();//5
\end{lstlisting}} \\
	local variable &
{\begin{lstlisting}[style=C++]
constintMAXSIZE=1024; 
char s[MAXSIZE];
\end{lstlisting}} &
{\begin{lstlisting}[style=C++]
string s;
\end{lstlisting}} \\
	copy &
{\begin{lstlisting}[style=C++]
strcpy(s, hello);
\end{lstlisting}} &
{\begin{lstlisting}[style=C++]
s = hello;
\end{lstlisting}} \\
	concatenate &
{\begin{lstlisting}[style=C++]
constchar* world = " world";
char message[MAXSIZE];
strcpy(message, hello);
strcat(message, world);
\end{lstlisting}} &
{\begin{lstlisting}[style=C++]
string message = hello + " world";
\end{lstlisting}} \\
	compare &
{\begin{lstlisting}[style=C++]
if (strcmp(a,b) == 0)
	// do something
\end{lstlisting}} &
{\begin{lstlisting}[style=C++]
if (a == b)
	// do something
\end{lstlisting}} \\
	convert to C++ string &
{\begin{lstlisting}[style=C++]
string cpp_str = hello;
\end{lstlisting}} &
{\begin{lstlisting}[style=C++]
char c_str[MAXSIZE];
strcpy(c_str, message.c_str());
\end{lstlisting}} \\
\end{tabular}
\end{center}

\subsection{Type Sizes}
\begin{itemize}
	\item The amount of memory assigned to a data type is a source of innumerable ``portability bugs'' in programs.
	\item There are \textbf{some} guarantees, however:
	\begin{itemize}
		\item A ``char'' is always one byte
		\item A ``short'' is always at least as big as a char
		\item An ``int'' is always at least as big as a short
		\item A ``long'' is always at least as big as an int
	\end{itemize}
	\item \lstinline[style=C++]{sizeof(int)} tells you the number of bytes required to store an \lstinline[style=C++]{int}
	\begin{center}
		\includegraphics{sections/lec7/type.png}
	\end{center}
\end{itemize}

\section{More recurrent neural networks}

\section{Ways to make neural networks generalize better}

\section{Combining multiple neural networks to improve generalization}

\section{Hopfield nets and Boltzmann machines}
	\subsection{Subclass}
\begin{itemize}
	\item In Flatland, soldiers and craftsmen have something in common: they are both workers, represented by triangles.
	\item We can represent this kind of relationship between two C++ classes with a \textit{derived class} (or referred to as: \textit{inherited class} or \textit{subclass})
	\item Subclasses contain all of the functionality and data of their parent class
	\item An ``is a'' relationship
	\item e.g.,
\begin{lstlisting}[style=C++]
class Isosceles: public Triangle {
	// ...
}
\end{lstlisting}
	\item Isosceles ``is a'' triangle
\end{itemize}

\subsection{Class hierarchy}
\begin{itemize}
	\item Derivation is often represented by a graph, where each vertex is a class, and each edge shows derivation
	\begin{center}
		\includegraphics{sections/lec11/hi.png}
	\end{center}
\end{itemize}

\subsection{Adding member functions}
\begin{itemize}
	\item Subclasses need to respect the abstraction of their parent classes
	\item Use getters and setters to deal with private variables
\end{itemize}

\subsection{Derived class constructors}
\begin{itemize}
	\item Constructors \textbf{are not} inherited.
	\item Constructors run automatically, \textbf{starting with the base class}.
	\item e.g.,
	\begin{itemize}
		\item First, \lstinline[style=C++]{Triangle} constructor runs
		\item Then, \lstinline[style=C++]{Isosceles} constructor runs
	\end{itemize}
	\item You can also re-use the parent class constructor:
\begin{lstlisting}[style=C++]
Isosceles(double base, double leg): Triangle (base, leg, leg) {}
\end{lstlisting}
	\item But do not call the parent constructor outside the initializer list
	\item \textbf{DO NOT DO THIS:}
\begin{lstlisting}[style=C++]
Isosceles(double base, double leg){
	Triangle(base, leg, leg); // BAD - anonymous object
}
\end{lstlisting}
\end{itemize}

\subsection{Override vs. Overload}
\begin{itemize}
	\item A function \textbf{override} is where a derived class has a function with the same name and prototype as the parent
\begin{lstlisting}[style=C++]
Triangle::set_b(double b_in);
Isosceles::set_b(double b_in);
\end{lstlisting}
	\item A function \textbf{overload} is where a single class has two different functions with the same name, but different prototypes
\begin{lstlisting}[style=C++]
Triangle::Triangle();
Triangle::Triangle(double a_in,double b_in,double c_in);
\end{lstlisting}
\end{itemize}

\subsection{protected members}
\begin{itemize}
	\item \lstinline[style=C++]{protected} members can be seen by all members of this class and any derived classes
\begin{lstlisting}[style=C++]
class Triangle {
public:
	//member functions ...
protected:
	//edge lengths represent a triangle
	double a,b,c;
};
\end{lstlisting}
	\item Use the scope resolution operator (::) to call inherited functions: \lstinline[style=C++]{Triangle::set_b(b_in);}
\end{itemize}

\subsection{Liskov Substitution Principle}
\begin{itemize}
	\item If S is a subtypeof T, then objects of type T may be replaced with objects of type S without altering any of the desirable properties of that program (correctness)
	\item Or: For any instance where an object of type T is expected, an object of type S can be supplied without changing the correctness of the original computation
	\item Not all C++ derived classes are subtypes
	\item For a derived type to also be a subtype, code written to correctly use the supertypemust still be correct if it uses the subtype
	\item 
\end{itemize}

\subsection{How to create a subtype}
\begin{itemize}
	\item With Abstract Data Types, there are three ways to create a subtype from a derived type:
	\begin{enumerate}
		\item Weaken the precondition of one or more operations
		\item Strengthen the postconditionof one or more operations
		\item Add one or more operations
	\end{enumerate}
	\item 1 \& 2 apply to overriden functions:
	\begin{itemize}
		\item The overridden member function must require no more of the caller than the old method did, but it can require less
		\item The overridden member function must do everything the old function did, but it is allowed to do more as well
	\end{itemize}
\end{itemize}

\subsubsection{Weaken precondition}
\begin{itemize}
	\item The preconditions of a method are formed by two things:
	\begin{itemize}
		\item Its argument type signature
		\item The REQUIRES clause
	\end{itemize}
	\item We can weaken the preconditions by requiring less:
\begin{lstlisting}[style=C++]
//REQUIRES: b_inis non-negative and forms a
// triangle with existing edges
//MODIFIES: this
//EFFECTS: sets edges b and c
void Isosceles::set_b(double b_in) {
	Triangle::set_b(b_in);
	Triangle::set_c(b_in);
}

//Becomes:
//REQUIRES: b_inis non-negative andforms a
// triangle with existing edges
//MODIFIES: this
//EFFECTS: sets edges b and c
void Isosceles::set_b(double b_in) {
	b_in= abs(b_in);
	Triangle::set_b(b_in);
	Triangle::set_c(b_in);
}
\end{lstlisting}
\end{itemize}

\subsubsection{Strengthen postcondition}
\begin{itemize}
	\item The postconditionsof a method are formed by two things:
	\begin{itemize}
		\item Its return type signature
		\item The EFFECTS clause
	\end{itemize}
	\item We can strengthen the EFFECTS clause by promising everything we used to, plus extra
\begin{lstlisting}[style=C++]
void Isosceles::set_b(double b_in) {
	Triangle::set_b(b_in); //does everything Triangle did
	Triangle::set_c(b_in); //plus more
}
\end{lstlisting}
\end{itemize}

\subsubsection{Add an operation}
\begin{itemize}
	\item The final way of creating a subtype is to add a member function
	\item Any code expecting only the old function will still see all of them, so the new function won't break old code
\begin{lstlisting}[style=C++]
class Isosceles : public Triangle {
public:
	//...
	void set_base(double base){/*...*/}
	void set_leg(double leg) {/*...*/}
};
\end{lstlisting}
\end{itemize}

\section{Restricted Boltzmann machines (RBMs)}
	\subsection{Boltzmann machine learning}
\begin{itemize}
	\item Unsupervised learning problem
	\item We want the maximize the product of the probabilities that the Boltzmann machine assigns to the binary vectors in the training set
	\item Equivalent to maximizing the sum of the log probabilities that the Boltzmann machine assigns to the training vectors
	\item It is also equivalent to maximizing hte probability that we would obtain exactly the N training cases if we did the following:
	\begin{itemize}
		\item Let the network settle to its stationary distribution N different times with no external input
		\item Sample the visible vector once each time
	\end{itemize}

	\subsubsection{Why the learning could be difficult}
	\item Consider the chain of units:
	\begin{center}
		\includegraphics[scale=0.6]{sections/12/chain.png}
	\end{center}
	\item If the raining set consists of (1,0) and (0,1) we want the product of all the weights to be negative (So to know how to change w1 or w5 we must know w3)

	\item Everything that one weight needs to know about the other weights and the data is contained in the difference of two correlations:
	\begin{center}
		\includegraphics[scale=0.6]{sections/12/corr.png}
	\end{center}
		$$\delta w_{ij} \propto \langle s_i s_j \rangle_{data} - \langle s_i s_j \rangle_{model}$$

	\subsubsection{Why is the derivative so simple?}
	\item THe probability of a global configuration at thermal equilibrium is an exponential function of its energy
	\item So setting to equilibrium makes the log probability a linear function of the energy
	\item The energy is a linear fnuction of the weights and states, so:
		$$-\frac{\partial E}{\partial w_{ij}}=s_i s_j$$

	\item The process of settling to thermal equilibrium propagates information about the weights (We dontt need backprop)

	\subsubsection{Why do we need the negative phase?}
	\item Probability of a visible vector:
	\begin{center}
		\includegraphics[scale=0.6]{sections/12/neg.png}
	\end{center}

	\subsubsection{An inefficient way to collect the statistics required for learning}
	\item \textbf{Positive phase}: clamp a data vector on the visible units and set the hidden units to random binary states
	\item Update the hidden units one at a time until the network reaches thermal equilibrium at a temperature of 1
	\item Once equilibrium, Sample $s_i s_j$ for every connected pair of units
	\item Repeat for al ldata vectors in the training set and average
	\item \textbf{Negative phase}: Set all the units to random states
	\item Update until equilibrium at temperature of 1
	\item Sample $s_i s_j$ for every connected pair of units
	\item Repeat many times and average to get good estimates
\end{itemize}

\section{Stacking RBMs to make Deep Belief Nets}
	\subsection{The ups and downs of backpropagation}
\begin{itemize}
	\subsubsection{A brief history of backpropagation}
	\item The backpropagation algorithm for learning multiple layers of features was invented several times in the 70's and 80's:
	\begin{itemize}
		\item Bryson \& Ho (1969) linear
		\item Werbos (1974)
		\item Rumelhart et. al. in 1981
		\item Parker (1985)
		\item LeCun (1985)
		\item Rumelhart et.al. (1985)
	\end{itemize}

	\item Backpropagation clearly had great promise for learning multiple layers of non-linear feature detectors
	\item But by the late 1990's most serious researchers in machine learning had given up on it
	\begin{itemize}
		\item It was still widely used in psychological models and in practical pplications such as credit card fraud detection
	\end{itemize}

	\subsubsection{Why backpropagation failed}
	\item The popular explanation of why backpropagation failed in the 90's:
	\begin{itemize}
		\item It could not make good use of multiple hidden layers
		\item It did not work well in recurrent networks or deep auto-encoders
		\item Support vector machines worked better, required less expertise, produced repeatable results, and had a much better theory
	\end{itemize}

	\item The real reason it failed:
	\begin{itemize}
		\item Computers were thousands of times too slow
		\item Labeled datasets were hundreds of times too small
		\item Deep networks were too small and not initialized sensibly
	\end{itemize}

	\subsubsection{A spectrum of machine learning tasks}
	\item Theres a spectrum of tasks from the things people study in typical statistics to artificial intelligence
	\item Statistics end of the spectrum:
	\begin{itemize}
		\item Low-dimensional data (e.g., less than 100 dimensions)
		\item Lots of noise in the data
		\item Not much structure in the data. The structure can be captured by a fairly simple model
		\item The main problem is separating true structure from noise
		\begin{itemize}
			\item Not ideal for non-Bayesian neural nets. Try SVM or GP.
		\end{itemize}
	\end{itemize}

	\item Artificial intellience end of the spectrum:
	\begin{itemize}
		\item High-dimensional data (e.g., more than 100 dimensions)
		\item The noise is not the main problem
		\item There is a huge amount of structure in the data, but its too complicated to be represented by a simple model
		\item The main problem is figuring out a way to represent the complicated structure so that it can be learned
		\begin{itemize}
			\item Let backpropagation figure it out
		\end{itemize}
	\end{itemize}

	\subsubsection{Why Support Vector Machines were never a good bet for Artificial Intelligence tasks that need good representations}
	\item \textbf{View 1}: SVM's are just a clever reincarnation of Perceptrons
	\begin{itemize}
		\item They expand the input to a very large layer of non-linear non-adaptive features 
		\item They only have one layer of adaptive weights 
		\item They ahve a very efficient way of fitting the weights that control overfitting (max margin hyperplane)
	\end{itemize}
	\item \textbf{View 2}: SVM's are just a clever reincarnation of Perceptrons (different notion of features being used)
	\begin{itemize}
		\item They use each input vector in the training set to define a non-adaptive ``feature''
		\begin{itemize}
			\item How similar a test input is to a training case
		\end{itemize}
		\item They have a clever way of simultaneously doing feature selection and finding weights on the remaining features.
	\end{itemize}
\end{itemize}

\subsection{Belief Nets}
\begin{itemize}
	\item 
\end{itemize}

\section{Deep neural nets with generative pre-training}

\section{Modeling hierarchical structure with neural nets}

\section{Recent applications of deep neural nets}

\end{document}