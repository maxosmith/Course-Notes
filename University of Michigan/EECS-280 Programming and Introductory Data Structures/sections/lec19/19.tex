\subsection{Introduction}
\begin{itemize}
	\item Abstract Data Types like \lstinline[style=C++]{IntSet} and \lstinline[style=C++]{IntList} are often called \textbf{containers} or container classes, since their purpose is to contain other objects
	\item Today, we will use the C++ \lstinline[style=C++]{template} mechanism to reuse the same container code for any type
	\item To see how templates might be useful, consider the following fragments defining a list-of-intand a list-of-char:
\begin{lstlisting}[style=C++]
// Int 
class List {
public:
	void insertFront(int v);
	int removeFront();
	//...
private:
	struct Node { 
		Node *next;
		int datum;
	};
	Node *first;
};

// Char 
class List {
public:
	void insertFront(char v);
	char removeFront();
	//...
private:
	struct Node { 
		Node *next;
		int datum;
	};
	char *first;
};
\end{lstlisting}
	\item It's like someone took the list-of-int definition and replaced each instance of \lstinline[style=C++]{int} with an instance of \lstinline[style=C++]{char}.
\end{itemize}

\subsection{Templates}
\begin{itemize}
	\item The intuition behind templates is that they are code with the ``type name'' left as a (compile-time) constant.
	\item The parameter is a type, not a variable.
	\item To start, you first need to declare that something will be a template:
\begin{lstlisting}[style=C++]
template <typename T>
class List {
	// ...
};
\end{lstlisting}
	\item \lstinline[style=C++]{T} stands for ``the name of the type contained by this \lstinline[style=C++]{List}''
	\item Alternate notation:
\begin{lstlisting}[style=C++]
template <class T>
class List {
	// ...
};
\end{lstlisting}
	\item e.g.,
\begin{lstlisting}[style=C++]
template <typename T>
class List {
public:
	// methods

	// constructors/destructor

private:
	struct Node {
		Node *next;
		T datum;
	};
	//...
};
\end{lstlisting}
\end{itemize}

\subsection{Templated Methods}
\begin{itemize}
	\item All that is left is to define each of the method bodies.
	\item Each body must also be declared as a templated method and we do that in much the same way as we do for the class definition
	\item Each function begins with the template declaration: \lstinline[style=C++]{template <typename T>}
	\item And each method name must in the \lstinline[style=C++]{List<T>} scope:
\begin{lstlisting}[style=C++]
template <typename T>
void List<T>::insertFront(Ti) {
	Node *np= new Node;
	np->datum = i;
	np->next = first;
	first = np;
}
\end{lstlisting}
\end{itemize}

\subsection{Using Templated Containers}
\begin{itemize}
	\item To use the templated container, you specify the type \lstinline[style=C++]{T} when creating the container object.
\begin{lstlisting}[style=C++]
// Create a list of integers on the stack
List<int> li;
li.insertFront(42);

// Create a list of integers on the heap
List<int> *lip = new List<int>;
lip->insertFront(42);
\end{lstlisting}
	\item All templated codes goes in the header file: both the declaration and definition
\end{itemize}

\subsection{Containers of large values}
\begin{itemize}
	\item Copying elements by value is fine for types with small representations, for example, built-in types like \lstinline[style=C++]{int}
	\item This is nottrue for larger types –any nontrivial structor class would be expensive to pass by value, because you'll spend all of your time copying
	\item e.g.,
\begin{lstlisting}[style=C++]
class Gorilla {
	// OVERVIEW: a big, expensive class
	// lots of member data ...
};\end{lstlisting}
	\item Two options:
	\begin{itemize}
		\item Pass-by-pointer (results in container of pointers)
		\item Pass-by-reference (only fixes parameter copy issues)
	\end{itemize}
\end{itemize}

\subsubsection{Container of pointers}
\begin{itemize}
	\item The good news: \lstinline[style=C++]{Gorilla} is passed by pointer, no copies!
	\item The bad news: containers-of-pointers are subject to two broad kinds of potential bugs:
	\begin{enumerate}
		\item Using an object after it has been deleted
\begin{lstlisting}[style=C++]
List<Gorilla*> zoo;
Gorilla *colo= new Gorilla;
zoo.insertFront(colo);
// ...
delete colo; // bad!
// ...
Gorilla *g = zoo.removeFront();//g already deleted!
\end{lstlisting}
		\item Leaving an object orphaned by neverdeleting it
	\end{enumerate}
\end{itemize}

\subsection{Pattern of use for container-of-ptr}
\begin{enumerate}
	\item \textbf{Existence}: Allocate an object before inserting it into any container.
\begin{lstlisting}[style=C++]
// OK
List<Gorilla*> zoo;
Gorilla *colo= new Gorilla;
zoo.insertFront(colo);

// better
List<Gorilla*> zoo;
zoo.insertFront(new Gorilla);
\end{lstlisting}
	\item \textbf{Ownership}: Once it is inserted, do not modify it until it is removed.
\begin{lstlisting}[style=C++]
List<Gorilla*> zoo;
Gorilla *colo= new Gorilla;
zoo.insertFront(colo);
colo->set_name("Bill"); // bad! - could mess up a container's representation invariant

// Good:
Gorilla *g = zoo.removeFront(); //remove
g->set_name("Bill"); //change
zoo.insertFront(g); //replace
\end{lstlisting}
	\item \textbf{Conservation}: Once it is removed, it must either be deallocatedor inserted into some container. Don’t forget to delete at the end!
\begin{lstlisting}[style=C++]
// bad example
List<Gorilla*> zoo;
zoo.insertFront(new Gorilla);
zoo.removeFront(); //bad! memory leak!

// good example
List<Gorilla*> zoo;
zoo.insertFront(new Gorilla);
deletezoo.removeFront(); // fixed
\end{lstlisting}
\end{enumerate}

\subsection{Deleting a shared object}
\begin{itemize}
	\item How to delete objects that occur on multiple containers?
	\item One solution:
	\begin{itemize}
		\item Remove object from allcontainers before freeing up the object.
	\end{itemize}
\end{itemize}