\subsection{Types of testing}
\begin{itemize}
	\item Unit testing
	\begin{itemize}
		\item Test smaller, less complex, easier to understand units
		\item One piece at a time (e.g., a function)
	\end{itemize}

	\item System testing
	\begin{itemize}
		\item Test entire project (code base)
		\item Do this \textit{after} unit testing
	\end{itemize}

	\item Regression testing
	\begin{itemize}
		\item Automatically run all unit and system tests after a code change
	\end{itemize}

	\item Steps to test:
	\begin{enumerate}
		\item Understand the specification
		\item Write tests
		\item Check the results
	\end{enumerate}
\end{itemize}

\subsection{Understand the Specification}
\begin{itemize}
	\item For an entire assignment, read through the specification very carefully, and make a note of everything it says you have to do
	\item Required behaviors: what must and must not happen
\end{itemize}

\subsection{Write Tests}
\begin{itemize}
	\item For each of your required behaviors, write one or more test cases that check them.
	\item To the extent possible, the test case should check exactlyone behavior –no more!
	\item 4 types of tests:
	\begin{itemize}
		\item \textbf{Simple}: ``expected'' or ``normal'' inputs
		\item \textbf{Boundary}: edges of what is expected
		\item \textbf{Nonsense}: unexpected inputs
		\item \textbf{Stress}: long inputs
	\end{itemize}
\end{itemize}

\subsection{Check the results}
\begin{itemize}
	\item Do: write down correct answer beforerunning test
	\item Don't: quickly run test cases and glance at the output
	\item \textbf{Asserts} that the representation invariant is true:
\begin{lstlisting}[style=C++]
#include <cassert>
list_tin = list_make(); // empty list
list_tout = reverse(in); // expect empty list
assert( list_isEmpty(out) );
\end{lstlisting}
\end{itemize}

\subsection{Debugging Strategies}
\begin{itemize}
	\item Read the code
	\item Remove portions of the code to narrow down error
	\item Add print statements
	\item Add \lstinline[style=C++]{assert()} statements
	\item Add a more specific test (shorter is better)
	\item \textit{Use a debugger!}
\end{itemize}

\subsection{Function Pointers}
\begin{itemize}
	\item How do you define a variable that points to a function taking two integers, and returns an integer?
\begin{lstlisting}[style=C++]
int (*foo)(int, int);
\end{lstlisting}
	\item You read this from ``inside out'':
\begin{lstlisting}[style=C++]
foo						// foo
(*foo)					// is a pointer
(*foo)(    );			// to a function
(*foo)(int, int);		// that takes two integers
int (*foo)(int, int);	// and returns an integer
\end{lstlisting}
	\item Once we've declared foo, we can assign any function to it: \lstinline[style=C++]{foo = min;}
	\item And then it can be called as follows: \lstinline[style=C++]{foo(3, 5)}
\end{itemize}