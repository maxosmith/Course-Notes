\subsection{Tail-recursion to iteration conversion}
\begin{itemize}
	\item There are five steps to the conversion of a tail-recursive function to an iterative one:
	\begin{enumerate}
		\item Copy the function's type signature
		\item Identify any needed ``loop variables'' by inspecting the call to the helper function (if it exists).
		\item Write initialization code to mirror the call to the helper function
		\item Identify termination condition(s) and return values by copying the base case behavior.
		\item Write loop body by copying the inductive step
	\end{enumerate}
\end{itemize}

\subsection{Dependency Graphs}
\begin{itemize}
	\item Special kind of graph with ``directed'' edges showing which ``new'' values depend on which ``old'' values.
	\item An edge is drawn from a \textbf{source vertex} to a \textbf{sink vertex}.
	\item To build a dependency graph, draw one vertex for each variable. If variable fooreads from variable barto compute its new value, draw an edge fromfootobar. (Don't draw an edge between a vertex and itself).
	\item If a variable has no edges with it as a sink(i.e. no edges terminate there), you can write its assignment, and erase it and any edges with it as a source.
	\item If two variables (\lstinline[style=C++]{foo, bar}) are each dependent on the other, then you can solve this by inventing shadow variables:
\begin{lstlisting}[style=C++]
int foo_new = bar - 1; // Shadow variable
int bar_new = foo - 1; // Shadow variable
// ------------------
foo = bar_new;
bar = foo_new;
\end{lstlisting}
	\item The transition between these two steps is called a \textbf{software epoch} - dependencies do not exist across epochs.
\end{itemize}