\subsection{Abstraction}
\begin{itemize}
	\item \textbf{Abstraction} is a many-to-one mapping that reduces complexity and eliminates unnecessary details by providing only those details that matter.
	\item e.g., Multiplication: table look-up, summing, etc. (we only care that it multiplies)
	\item Type types: procedural, data
\end{itemize}

\subsection{Procedural Abstraction}
\begin{itemize}
	\item Decomposing programs into functions
	\item For any function, there is a person who implements the function (the author) and a person who uses the function (the client).
	\item The author needs to think carefully about whatthe function is supposed to do, as well as howthe function is going to do it.
	\item In contrast, the client only needs to consider the what, not the how. Since how is much more complicated, this is a Big Win for the client!
	\item Provides two important properties:
	\begin{itemize}
		\item \textbf{Local}: the implementation of an abstraction can be understood without examining any other abstraction implementation.
		\item \textbf{Substitutable}: you can replace one (correct) implementation of an abstraction with another (correct) one, and no callers of that abstraction will need to be modified.
	\end{itemize}
	\item Layout abstractions before writing code, you don't want to change it!
	\item If a change can be limited to replacing the implementation of an abstraction with a substitutable implementation, then you are guaranteed that no other part of the project needs to change.
\end{itemize}

\subsection{Fucntion Definitions vs. Declarations}
\begin{itemize}
	\item In larger programs, it is useful to separate the declarationof a function from its definition. A declaration provides the \textbf{type signature} of a function, also called the function header.
	\item Function headers can be placed in their own file and accessed using the preprocessor directive \lstinline[style=bash]{#include}.
	\item The \textbf{header file} contains all of the function declarations (io.h):
\begin{lstlisting}[style=C++]
double GetParam(string prompt, double min, double max);
//...
\end{lstlisting}
	
	\item And the associated implementations/definitions (io.cpp):
\begin{lstlisting}[style=C++]
#include "io.h"
double GetParam(string prompt, double min, double max){
	/* ... */
}
\end{lstlisting}

	\item These can be accessed in another file (p1.cpp) through an include of the header file:
\begin{lstlisting}[style=C++]
#include "io.h"
int main() {
	//...
	GetParam(prompt, min, max);
}
\end{lstlisting}
\end{itemize}

\subsection{Function Details}
\begin{itemize}
	\item All C++ functions take zero or more arguments and return a result of some type.
	\item There is a special type, called \textbf{void}, that means ``noresult is returned''. void is still a type, even though it is ``the type with no legal values''.
	\item The type signature of a function can be considered part of theabstraction –if you change it, customers (callers) must also change.
	\item However, as long as a new implementation of an abstraction does the ``same thing'' as some old (correct) implementation, you can replace the old one with the new one.
	\item Now we need to define what ``same thing'' means through \textbf{specifications}
\end{itemize}

\subsection{Specifications}
\begin{itemize}
	\item Answer three questions:
	\begin{itemize}
		\item \textbf{REQUIRES}: What pre-conditions must hold? (if any)
		\item \textbf{MODIFIES}: Do inputs change? How? (if any)
		\item \textbf{EFFECTS}: What does it do?
	\end{itemize}

	\item e.g.,
\begin{lstlisting}[style=C++]
intfactorial(intn);
// REQUIRES: n >= 0
// EFFECTS: returns n!
\end{lstlisting}

	\item Functions without REQUIRES clauses are considered \textbf{complete}; they are valid for all input. Otherwise they're \textbf{partial}.
	\item Complete is better to write than partial, if possible.
	\item A MODIFIES clause identifies any function argument or piece of global state that \textbf{might} change if this function is called.
	\item e.g.,
\begin{lstlisting}[style=C++]
void swap(int&x, int&y);
// MODIFIES: x, y
// EFFECTS: exchanges the values of x and y
\end{lstlisting}
	\item The ampersand (\&) means that you get a \textbf{reference} to the caller's argument, not a cop yof it. This lets you to change the value of that argument.
\end{itemize}

\subsection{Call Stacks: How function calls work}
\begin{itemize}
	\item When you call a function, the program follows these steps:
	\begin{enumerate}
		\item Evaluate the actual arguments to the function (order is not guaranteed).
		\item Create an ``activation record'' (sometimes called a ``stack frame'') to hold the function's formal arguments and local variables.
		\item Copy the actuals' values to the formals' storage space.
		\item Evaluate the function in its local scope.
		\item Replace the function call with the result.
		\item Destroy the activation record.
	\end{enumerate}

	\item Activation records are typically stored as a \textbf{stack}.
	\item See example from lecture slides.
\end{itemize}

\subsection{Recursion}
\begin{itemize}
	\item \textbf{Recursive}: ``refers to itself''
	\item A function is recursive if it calls itself
	\item Problem is recursive if:
	\begin{enumerate}
		\item Base case exists (at least one)
		\item All other cases can be solved by first solving one (or more) smaller cases, and then combining those solutions with a simple step.
	\end{enumerate}
	\item e.g.,
	$$n!=\begin{cases}
		1 &(n==0)\\
		n\times (n-1)! &(n>0)
	\end{cases}$$
\begin{lstlisting}[style=C++]
int factorial(int n) {
	// REQUIRES: n >= 0
	// EFFECTS: computes n!
	if (n == 0){
		return 1; // Base case
	}else {
		return n * factorail(n-1); // Recursive step
	}
}
\end{lstlisting}
	
	\item To write a recursive function:
	\begin{enumerate}
		\item Identify the ``trivial'' cases, and write the explicitly
		\item For all other cases, first assume there is a function that can solve smaller versions of the same problem, then figure out how to get from the smaller solution to the bigger one.
	\end{enumerate}
	\item See example from lecture slides.
\end{itemize}