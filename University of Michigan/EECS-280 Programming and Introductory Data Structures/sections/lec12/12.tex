\subsection{Static type}
\begin{itemize}
	\item \textbf{Static type}: when a compiler can tell which derived type is needed
\begin{lstlisting}[style=C++]
Trianglet(3,4,5);
t.print();

Triangle *t_ptr= &t;
t_ptr->print();

Isoscelesi(1,12);
i.print();

Isosceles*i_ptr= &i;
i_ptr->print();
\end{lstlisting}
\end{itemize}

\subsection{Dynamic type}
\begin{itemize}
	\item \textbf{Dynamic type}: type is not known until run time 
\begin{lstlisting}[style=C++]
//EFFECTS: asks user to select Triangle,
// Isosceles or Equilateral
// returns a pointer to correct object
Triangle * ask_user();

int main() {
	Triangle *t = ask_user(); //enters "Isosceles"
	t->print();
}
\end{lstlisting}
\end{itemize}

\subsection{Polymorphism}
\begin{itemize}
	\item From the previous example:
\begin{lstlisting}[style=C++]
//...
Triangle *t = ask_user(); //"Isosceles"
t->print();
\end{lstlisting}
	\item \lstinline[style=C++]{t} can change types at runtime, in other words it is \textbf{polymorphic}
	\item Polymorphism is the ability to associate many behaviors with one function name dynamically (at runtime)
\end{itemize}

\subsection{Virtual Functions}
\begin{itemize}
	\item A \lstinline[style=C++]{virtual} function checks dynamtic type to find correct overridden function version to run.
\begin{lstlisting}[style=C++]
class Triangle {
	virtual void print() const{/*...*/}
	//...
};
\end{lstlisting}
	\item The \lstinline[style=C++]{virtual} function type is inherited, so any sub-classes of \lstinline[style=C++]{Triangle} that override \lstinline[style=C++]{print()} will automatically become \lstinline[style=C++]{virtual}
\end{itemize}

\subsection{Abstract base class}
\begin{itemize}
	\item An \textbf{abstract base class} creates an \textit{interface only} class as a base class, from which an implementation can be derived
	\item You \textbf{cannot create an instance} of an abstract base class.
	\item In C++ a class is abstract if there's missing/no implementation (via having at least one of its functions as \textbf{pure virtual} function)
\begin{lstlisting}[style=C++]
class Shape {
public:
	virtual double area() const = 0;
	virtual void print() const = 0;
};
\end{lstlisting}
	\item ``Assigning'' a 0 to a function designates as pure virtual
	\item A \textbf{factory function} creates objects for a client who doesn't need to know their actual types (dynamic)
\end{itemize}

\subsection{Upcast}
\begin{itemize}
	\item An \textbf{upcast} is substituting a subtype for a supertype
	\item Type conversion is automatic, an \textit{implicit cast}
\begin{lstlisting}[style=C++]
Shape * ask_user() {
	cout << "Rectangle, Triangle, Isosceles or Equilateral? ";
	string s;
	cin >> s;
	if (s == "Triangle") return &g_triangle;	// Upcasted to shape
	if (s == "Isosceles") return &g_isosceles;	// Upcasted to shape
	//...
}
\end{lstlisting}
	\item Since \lstinline[style=C++]{Isosceles} \textit{is a} \lstinline[style=C++]{Shape}, this can happen automatically.
\end{itemize}

\subsection{Downcast}
\begin{itemize}
	\item Can't convert supertype to subtype
	\item e.g., \lstinline[style=C++]{Shape}$\to$\lstinline[style=C++]{Isoscelese}
	\item Type conversion is not automatic
	\item This is called a \textit{downcast}
\begin{lstlisting}[style=C++]
Isosceles *t= ask_user();//enter "Isosceles"
// Error: invalid conversion
\end{lstlisting}
	\item Since a parent type might not be a child type, this cast cannot happen automatically
	\item \lstinline[style=C++]{dynamic_cast<T*>(ptr)} downcasts \lstinline[style=C++]{ptr} to type \lstinline[style=C++]{T*}
\end{itemize}

\subsection{dynamic\_cast vs. static\_cast}
\begin{center}
\begin{tabular}{l|l}
	dynamic\_cast & static\_cast \\
	\hline
	Checks at runtime if it is OK to cast & Does not check at runtime \\
	Cast from a polymorphic base to derived & Programmer tells compiler ``trust me'' \\
	Supertype to subtype, downcast & Non-polymorphic types \& polymorphic \\
\end{tabular}
\end{center}