\subsection{Abstraction}
\begin{itemize}
	\item \textbf{Procedural abstraction}
	\begin{itemize}
		\item A function is given a specification which documents whatthe function does, but not howit does it.
		\item For example, if we find a faster way to compute factorial, we can just replace the old implementation with the new one, and no other component of the program needs to know this.
	\end{itemize}
	\item \textbf{Data abstraction}
	\begin{itemize}
		\item An ADT provides an abstractdescription of values and operations.
		\item The definition of an ADT must combine bothsome notion ofwhatthe values of that type represent, and howthey behave.
		\item We can leave off the details of how this actually happens.
	\end{itemize}
\end{itemize}

\subsection{Abstract Data Types}
\begin{itemize}
	\item \textbf{Information hiding}: we don't need to know the details of how the object is represented, nor do we need to know how the operations on those objects are implemented.
	\item \textbf{Encapsulation}: the objects and their operations are defined in the same place; the ADT combines both data and operation in one entity.
	\item ADTs are substitutable: you can change the implementation and no users of the type can tell
	\item See example in lecture slides.
\end{itemize}

\subsection{const member functions}
\begin{itemize}
	\item You can use \lstinline[style=C++]{const} in a function signature promises ``this member function will not modify any member variable''
\begin{lstlisting}[style=C++]
class Triangle {
	//...
	double area() const;
	//...
}
\end{lstlisting}
\end{itemize}

\subsection{Representation invariant}
\begin{itemize}
	\item Member variables are a class’s representation
	\item The description of how member variables should behave are representation invariants
	\item \textbf{Representation invariants} are rules that the representation must obey immediately before and immediately after any member function execution
	\item The \textit{what} the data type does is then the header file
	\item The \textit{how} the data type works is in the cpp file
\end{itemize}

\subsection{Scope Resolution Operator}
\begin{itemize}
	\item \lstinline[style=C++]{::} is the scope resolution operator, which tells the compiler that this function is inside the scope of the class
	\item e.g.,
\begin{lstlisting}[style=C++]
Triangle::Triangle(): a(0), b(0), c(0) {}
\end{lstlisting}
	\item In the previous example, it's used to show in-line construction
	\item This syntax is called an \textbf{initializer list}
	\item More efficient, because dealt with at compile time.
	\item It can also be used with parameters. e.g.,
\begin{lstlisting}[style=C++]
Triangle::Triangle(double a_in, double b_in, double c_in): a(a_in), b(b_in), c(c_in) {}
\end{lstlisting}
	\item See the Flatland example from lecture slides.
\end{itemize}