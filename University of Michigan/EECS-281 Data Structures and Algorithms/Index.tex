\documentclass[english, 11pt]{article}
\usepackage{../../notes}
\usepackage{lipsum}

%Global Course Variables
\newcommand{\myCourseCode}{EECS 281}
\newcommand{\myCourseName}{Data Structures and Algorithms}
\newcommand{\myProf}{David Paoletti}
\newcommand{\myTerm}{Winter 2015}
\newcommand{\myLogo}{../um_seal.png}

%Headers
\lhead{\myCourseName}
\rhead{\fancyplain{}{\rightmark}} 

%Footers
\cfoot{\thepage}

\begin{document}
\titleHeader{\myCourseCode}{\myCourseName}{\myProf}{\myTerm}{\myLogo}

%Document information
\noindent\rule{1\columnwidth}{.5pt}
Contributors: Steven Schmatz, Max Smith
\begin{center}
	Latest revision: \today
\end{center}
\toc
\abstr{EECS 281 is an introductory course in data structures and algorithms at the undergraduate level. The objective of the course is to present a number of fundamental techniques to solve common programming problems. For each of these problems, we will determine an abstract specification for a solution and examine one or more potential representations to implement the abstract specification, focusing on those with significant advantages in time/space required to solve large problem instances. When appropriate, we will consider special cases of a general problem that admit particularly elegant solutions.}

%----------------------------
%Document Begins
%----------------------------

\section{Introduction and Workflow}

\section{Complexity Analysis}
	\subsection{Notation}
\begin{itemize}
	\item $x\in\mathbb{R}^D$: data 
	\item $\phi (x)\in\mathbb{R}^M$: features for $\vec{x}$
	\item $t\in\mathbb{R}$: continuous-valued labels
	\item $\vec{x}^{(n)}\equiv \vec{x}_n$: n-th training example
	\item $\vec{t}^{(n)}\equiv \vec{t}_n$: n-th targe value
\end{itemize}

\subsection{1D Inputs}
\begin{itemize}
	\item In the 1D case ($x\in\mathbb{R}^1$)
	\item Given a set of observations $x^{(1)},\ldots ,x^{(N)}$ and corresponding targe values $t^{(1)},\ldots,t^{(N)}$
	\item We want to learn a function $y(\vec{x}, W)\approx t$ to predict future values
	$$y(\vec{x}, \vec{w})=\sum_{j=0}^{M-1}\vec{w}_j \phi_j (\vec{x}) = \vec{w}^T \phi(x)$$
	\item e.g., (green = solution, red = 3-rd polynomial approximation)
	\begin{center}
		\includegraphics{sections/lec2/3.png}
	\end{center}
	\item For simplicity, we add a \textit{bias function}: $\phi_0 (\vec{x})=1$
	$$\vec{\phi}=1, x, x^2, x^3, \ldots$$
\end{itemize}

\subsection{Basis Function}
\begin{itemize}
	\item Function to construct features from raw data.
	\item e.g.,
	\begin{itemize}
		\item Polynomial: $\phi_j(x)=x^j$
		\item Gaussian: $\phi_j(x)=exp(-\frac{(x-\mu_j)^2}{2s^2})$
		\item Sigmoid: $\phi_j(x)=\sigma (\frac{x-\mu_j}{s})$
	\end{itemize}
\end{itemize}

\subsection{Objective Function}
\begin{itemize}
	\item We will use of sum-of-square errors:
	$$E(w)=\frac{1}{2}\sum_{n=1}^N (y(x^{(n)}, w)-t^{(n)})^2$$
\end{itemize}

\subsection{Batch Gradient Descent}
\begin{itemize}
	\item Given data $(x,y)$ initial $w$, repeat until convergence:
	$$\vec{w}=\vec{w}-\eta \nabla_{\vec{w}} E(\vec{w})$$
	$$\nabla_{\vec{w}}E(w)=\sum_{n=1}^N \left( \sum_{k=0}^{M-1} w_k \phi_k (\vec{x}^{(n)}) - t^{(n)} \right)\phi (\vec{x}^{(n)})=\sum_{n=1}^N \left(\vec{w}^T \phi (\vec{x}^{(n)}) - \vec{t}^{(n)}\right) \phi(x^{(n)})$$
\end{itemize}

\subsection{Overfitting}
\begin{itemize}
	\item An implicit way to tell is when the coeffecients become unreasonably large
	\item Solutions:
	\begin{itemize}
		\item Reduce order
		\item Add more data point
		\item Reselect features, some may be harming you
	\end{itemize}
	\item If you have a small number of data points, then you should use low order polynomial (small number of features)
	\item As you obtain more data points, you can gradually increase the order of the polynomial (more features)
	\item Controlling model complexity: \textbf{regularization}
\end{itemize}

\section{Measuring Runtime, and Pseudocode}
	\subsection{Traceroute}
\begin{itemize}
	\item \lstinline[style=bash]{traceroute} uses debug messages to expose packet's route to destination
	\img{.8}{sections/lec3/tr.png}
	\item No reason to have the packet links to make sense (they're not based on geography, only connectivity)
	\item e.g.,
	\begin{itemize}
		\item Ann Arbor, MI
		\item San Mateo, CA
		\item New York, NY
		\item <ends>
	\end{itemize}
\end{itemize}

\subsection{IP's Small Corners}
\begin{itemize}
	\item How does a host get an IP address
	\begin{itemize}
		\item Used to be static
		\item Nowadays, often dynamic - DHCP (Dynamic Host Configuration Protocol)
	\end{itemize}
	\item How does a host get its gateway address (to chat from local to internet)?
	\begin{itemize}
		\item Dynamically from Router 
		\item Originally static each get their own wire
	\end{itemize}
	\subsection{ARP}
	\img{.8}{sections/lec3/arp.png}
	\item How does an IP address get mapped to a target ethernet address?
	\begin{itemize}
		\item Address Resolution Protocol (ARP)
	\end{itemize}
	\item This leaves the possibility of ARP spoofing (or poisoning) where you pretend to be another address
	\item \textbf{ARP spoofing}
	\begin{itemize}
		\item Associate attacker's MAC address with IP of another host(s)
		\item Man-in-the-middle attacks
		\item Reconfigures table switch
		\item Switch only sends to destination
		\item Hub sends to everyone
		\item CPU ignores not intended for your CPU (spoofing overrides this)
	\end{itemize}
\end{itemize}

\subsection{Domain Name Service}
\begin{itemize}
	\item How does a browser known which IP address to contact?
	\item BY a directory look-up, using \textbf{Domain Name Service (DNS)}
	\item Cache recently used domain names to reduce need for DNS look up
	\item e.g.,
	\begin{lstlisting}[style=bash]
$ host umich.edu
umich.edu has address 141.211.243.44
	\end{lstlisting}
	\item Host checks for name mapping in DNS (umich$\to$IP)
\end{itemize}

\subsection{TCP}
\begin{itemize}
	\item Transport Control Protocol
	\item Reliable, ordered bytestreams
	\item Connection-oriented, unlike IP
	\item ``Virtual circuit'' networking built on packet infrastructure
	\item Looks like a circuit, but no reservations (on IP)
	\item Processes tied to ``host:port'' pairs
	\begin{itemize}
		\item Packets/processes talk to ports
		\item Ports are OS-level concept
		\item Server ports are ``well0-known'' and associated with services; other ports are ``ephemeral''
	\end{itemize}
	\item Common TCP ports:
	\begin{itemize}
		\item HTTP - 80
		\item SSH - 22
		\item HTTPS - 442
		\item SMTP - 25
		\item IMAP - 143
	\end{itemize}
	\item Message stream is bidirectional
\end{itemize}

\subsection{Implementing Connections}
\begin{itemize}
	\item Basic principle of reliable TCP connection is retransmission
	\begin{itemize}
		\item Each packet has 32-bit sequence number
		\item Every SeqNo is ACKed (acknowledged) by receiver
		\item When timeout expires, sender emits again
	\end{itemize}
	\subsection{Stop and Wait}
	\img{.8}{sections/lec3/sw.png}
	\item Send a packet, wait for ACK
	\item Receiver ACKS everything
	\item Are there downsides to Stop-and-wait?
	\begin{itemize}
		\item Never get a resposne back
	\end{itemize}
\end{itemize}

\subsection{Sliding Window}
\begin{itemize}
	\item \textbf{Sliding window} technique for managing send/receive capacity
	\img{.8}{sections/lec3/sliding.png}
	\begin{itemize}
		\item Receiver indicates ``receive window'' it is willing to buffer
		\item Sender cannot have more packets in transit (unACKed) than what can fit in the window
		\item Ideally, window is bandwidth * RT delay
		\begin{itemize}
			\item Keeps as much data in transit as possible
		\end{itemize}
	\end{itemize}
	\item This algorithm has several roles:
	\begin{itemize}
		\item Reliable delivery, via ACKs, timeouts, and retransmissions
		\item In-order delivery, via buffering and ACKing
		\item Flow control, via advertised window for receiver
	\end{itemize}
	\item Problems with sliding window?
	\begin{itemize}
		\item What if sender trnasmits data too fast?
		\item Or receiver reads data too slow?
	\end{itemize}
	\subsection{Sliding Window - Sender}
	\img{.9}{sections/lec3/send.png}
	\item Sender buffers unACKed data
	\item Only removes data from send buffer after it's been ACKed
	\item Send window determined by receiver's advertisements
	\subsection{Sliding Window - Receiver}
	\img{.9}{sections/lec3/rec.png}
	\item Receiver buffers data that is 
	\begin{itemize}
		\item Out-of-order
		\item Not yet read off by application
	\end{itemize}
	\item ACKs data as it arrives
	\item Removes data from buffer as app reads
	\item Also shrinks/expands advertised window in response to application behavior
	\subsection{Example}
	\img{.8}{sections/lec3/w.png}
	\subsection{Errors}
	\img{.8}{sections/lec3/err.png}
	\item \textbf{ACK lost}: ACK was lost from receiver to sender, so the sender doesn't know that the receiver got the packet. Sends it again anyways!
	\item \textbf{Packet lost}: Receiver never got packet, so sender never gets ACK; therefore, it tries again!
	\item \textbf{Early timeout}: If the timeout is too short, the sender will send the same packet again before the ACK had time to come back. This just results in a waste of resourdces
\end{itemize}

\subsection{3-way Handshake}
\img{.8}{sections/lec3/3way.png}

\subsection{Connection Teardown}
\begin{itemize}
	\item Orderly release by sender + receiver
	\item ``Hanging up the phone''
	\item Releases states and buffers on both sides
	\img{.8}{sections/lec3/close.png}
\end{itemize}

\subsection{Slowing Down}
\begin{itemize}
	\item Two reasons for sender to slow down:
	\begin{itemize}
		\item Reeiver can't handle input (and will have to drop packets)
		\item Network can't transmit sas fast as incoming packets arrive
	\end{itemize}
	\item Sliding window algorithm takes care of the receiver
	\item Sender never transmits more than advertised window size 
\end{itemize}

\subsection{Congestion}
\begin{itemize}
	\item Buffers in middle of network can become overloaded
	\item Not part of window negotiation (not accounted for)
	\img{1}{sections/lec3/cong.png}
	\item When has a packet been lost?
	\begin{itemize}
		\item Too long a timer, you're waiting pointlessly
		\item Too short, adds needless load
	\end{itemize}
	\item Many retransmits can induce \textbf{congestion collapse}
	\begin{itemize}
		\item Retransmits just add to congestion
		\item Capacity of network falls dramatically
		\item Increase in load that results in a decrease in useful work done
	\end{itemize}
	\subsubsection{Where It Happens: Links}
	\item Simple resource allocation: FIFO queue and drop-tail
	\begin{itemize}
		\item Access to the bandwidth: FIFO queue
		\begin{itemize}
			\item Packets transmitted in the order they arrive
			\img{.7}{sections/lec3/fifo.png}
		\end{itemize}
		\item Access to the buffer space: drop-tail queueing
		\begin{itemize}
			\item If the queue is full, drop the incoming packet
			\img{.7}{sections/lec3/drop.png}
		\end{itemize}
	\end{itemize}
	\subsubsection{End Host}
	\item Packet delay: packet experiences high delay
	\item Packet loss: packet gets dropped along the way
	\item How does TCP sender learn this?
	\begin{itemize}
		\item Delay: round-trip time estimate
		\item Loss: timeou, acknowledgements
	\end{itemize}
\end{itemize}

\subsection{TCP Congestion Window}
\begin{itemize}
	\item Each TCP sender maintains a congestion window
	\begin{itemize}
		\item Maximum number of bytes to have in transit
		\item Number of bytes still awaiting acknowledgments
	\end{itemize}
	\item Adapting the congestion window
	\begin{itemize}
		\item Decrease upon losing a packet (backing off)
		\item Increase upon success (optimistically explorining)
		\item Always struggling to find the right transfer rate
	\end{itemize}
\end{itemize}

\subsection{Additive Increase Multiplicative Decrease (AIMD)}
\begin{itemize}
	\item AIMD algorithm:
	\begin{enumerate}
		\item After packet timeout, cwnd = cwnd/2
		\item If none, cwnd += 1 every RTT
		\item Sender always xmits min(win, cwnd)
	\end{enumerate}
	\item Why use AIMD? Network load is really hard to get rid of: oversubscribed link must be undersubscribed to dissipate queue
	\begin{itemize}
		\item Over: packets dropped and retransmitted
		\item Under: somewhat lower throughput
	\end{itemize}
	\subsubsection{AIMD Sawtooth}
	\img{.7}{sections/lec3/saw.png}
	\item The previous diagram has a slow-start, so we can implement exponential window size to begin with to increase starting speed
	\img{.7}{sections/lec3/exp.png}
\end{itemize}

\subsection{HTTP 1.0}
\begin{itemize}
	\item Every HTML-embedded item requires a GET (index.html, ad.gif, logo.gif, etc.)
	\item Naive HTTP on TCP yields awful performance:
	\begin{itemize}
		\item Many TCP connections created roundtrip
		\item Lots of slow-start delays
	\end{itemize}
\end{itemize}

\subsection{HTTP 1.1}
\begin{itemize}
	\item Persistent Connections
	\begin{itemize}
		\item Server does not close the connection after sending the resposne
		\item Client can re-use it
		\begin{itemize}
			\item Especially improves small objects
			\item Makes parallel downloads difficult
		\end{itemize}
	\end{itemize}
	\item Pipelining
	\begin{itemize}
		\item Many HTTP requests can be ``live'' at once, on the same persistent connection
		\item Send lots of HTTP request at once, then get lots of answers
		\item Server replies in the order received
	\end{itemize}
	\item Caching
	\begin{itemize}
		\item GET + IF-MODIFIED-SINCE <timestamp>
		\item When combined with browser cache, eliminates a lot of unneeded data transfer
		\item Same number of roundtrips
		\item Lots of different caches possible
	\end{itemize}
\end{itemize}

\section{Recursion and the Master Theorem}
	\subsection{Tail-recursion to iteration conversion}
\begin{itemize}
	\item There are five steps to the conversion of a tail-recursive function to an iterative one:
	\begin{enumerate}
		\item Copy the function's type signature
		\item Identify any needed ``loop variables'' by inspecting the call to the helper function (if it exists).
		\item Write initialization code to mirror the call to the helper function
		\item Identify termination condition(s) and return values by copying the base case behavior.
		\item Write loop body by copying the inductive step
	\end{enumerate}
\end{itemize}

\subsection{Dependency Graphs}
\begin{itemize}
	\item Special kind of graph with ``directed'' edges showing which ``new'' values depend on which ``old'' values.
	\item An edge is drawn from a \textbf{source vertex} to a \textbf{sink vertex}.
	\item To build a dependency graph, draw one vertex for each variable. If variable fooreads from variable barto compute its new value, draw an edge fromfootobar. (Don't draw an edge between a vertex and itself).
	\item If a variable has no edges with it as a sink(i.e. no edges terminate there), you can write its assignment, and erase it and any edges with it as a source.
	\item If two variables (\lstinline[style=C++]{foo, bar}) are each dependent on the other, then you can solve this by inventing shadow variables:
\begin{lstlisting}[style=C++]
int foo_new = bar - 1; // Shadow variable
int bar_new = foo - 1; // Shadow variable
// ------------------
foo = bar_new;
bar = foo_new;
\end{lstlisting}
	\item The transition between these two steps is called a \textbf{software epoch} - dependencies do not exist across epochs.
\end{itemize}

\section{Arrays and Container Classes}
	\subsection{After calibration}
\img{.8}{sections/lec5/p.png}
\begin{itemize}
	\item Internal parameters $K$ are known
	\item $R, T$ are known - but these can only relate $C$ to the calibraiton rig.
	\item You can't estimate $P_i$ from the single image measurement of $p_i$ because it may be anywhere along the line in space.
	\item We also don't have any information in the image to tell us the scale of anything in the world.
\end{itemize}

\subsection{Recovering structure from a single view}
\begin{itemize}
	\item You can make assumptions to get relative sizes of objects in images
	\item Pick a reference plane in the scene
	\item Pick a reference direction (not parallel to the reference plane) in the scene
	\img{.8}{sections/lec5/ref.png}
	\subsubsection{Geometry}
	\img{.8}{sections/lec5/van.png}
	\item Under perspective projection, parallel lines in three-space project to convering lines in the image plane. The common point of intersection, perhaps at infinity, is called the \textbf{vanishing point}
	\begin{itemize}
		\item projection of a point at infinity (goes infinity far)
		\img{.8}{sections/lec5/vp.png}
		\item Any two parallel lines have the same vanishing point
		\item The ray from $C$ through $v$ point is parallel to the lines
		\item AN image may have more than one vanishing point
	\end{itemize}
	\item Two or more vanishing points from lines known to lie in a single 3D plane establish a \textbf{vanishing line}, which completely determines the orientation of the plane
	\img{.7}{sections/lec5/house.png}
\end{itemize}

\subsection{The Cross Ratio}
\img{.8}{sections/lec5/y.png}
\begin{itemize}
	\item $Y$ is desired height to measure
	\item Compute Y from image measurements
	\begin{itemize}
		\item You'll need more than vanishing points to compute this
	\end{itemize}
	\item \textbf{Projective Invariant}: something that does not change under projective transformations (including perspective projection)
	\img{.8}{sections/lec5/l.png}
	\item The \textbf{cross ratio} of 4 collinear points is projective invariant
	$$\frac{||P_3 - P_1||\cdot ||P_4 - P_2||}{||P_3 - P_2||\cdot ||P_4 - P_1||}$$
	\item You can permute the point ordering and it will remain true.
	\item Often called the fundamental invariant of projective geometry
	\img{.8}{sections/lec5/inf.png}
	\item When you consider the points at $\infty$, and the point where the camera would meet the ground plane $v_z$ you have enough points to use the cross ratio on our camera model.
	\item You need the same points in the world and camera system (can't mix and match)
	\item You must make some assumption to determine the relative sizes, typically assume $L$
\end{itemize}

\subsection{Horizon line}
\img{.7}{sections/lec5/hor.png}
\begin{itemize}
	\item Sets of parallel lines on the same plane lead to collinear vanishing points the line is called the \textbf{horizon}
	\img{.8}{sections/lec5/ex.png}
	\item When trying to recover the structure within the camera reference system, we can check if two linesare parallel or not
	\begin{itemize}
		\item If they do, recognize the horizon line
		\item Measure if hte 2 lines meet the horizon
		\item If they do, they are parallel in 3D
		\item Actual scale of scene is not recovered, only relative distances
	\end{itemize}
\end{itemize}

\subsection{Lines in a 2D plane}
$$ax+by+c=0; l=\begin{bmatrix}
	a\\b\\c
\end{bmatrix}$$
\subsubsection{Intersecting lines}
\img{1}{sections/lec5/intersect.png}
\begin{itemize}
	\item The intersection of lines can be calculated with: 
	$$x=l\times l'$$
\end{itemize}

\subsection{Stereo-view geometry}
\img{1}{sections/lec5/tri.png}
\begin{itemize}
	\item Two camera perspectives allow us to find position of objects through \textbf{triangulation}
	\begin{itemize}
		\item This requires a few knowns, $K_1, K_2, R, T$
	\end{itemize}
	\item Small inaccuracies with knowns can lead to situations where the lines may not intersect 
	\item Instead, we'll find where they came close enough
		$$d^2(x_1, M_1X)+d^2(x_2, M_2X)$$
		$$d:=\text{distance between two lines}$$
\end{itemize}

\subsection{Epipolar Geometry}
\img{.7}{sections/lec5/ep.png}
\begin{itemize}
	\item \textbf{Epipolar plane} (grey): intersections of baseline with image planes
	\item \textbf{Baseline} (orange): projectison of other camera center
	\item \textbf{Epipolar lines} (blue): vanishing points of camera motion direction
	\item This framepoint, allows us to consider if two pointsare related by epipolar geometry
	\item Use epipolar lines to find sharing points will greatly save computation cost as a 2D search becomes 1D
	\subsubsection{Parallel image planes}
	\item Baseline intersects the image plane at infinity
	\item Epipoles are at infinity
	\item Epipolar lines are parallel to x-axis
	\subsubsection{Forward translation}
	\img{1}{sections/lec5/f.png}
	\item When a camera moves forward the lines turn into a spiral
\end{itemize}

\section{Linked Lists and Iterators}
	\subsection{Encryption as a Function}
\begin{itemize}
	\item Plaintest string $s$
	\item Encryption key $K_{enc}$
	\item Decryption key $K_{dec}$
	\item Encrypt $s$ with $K_{enc}$ to obtain ciphertext $K_{enc}(s)$
	\item Decrypt $K_{enc}(s)$ with decryption key $K_{dec}$ to reobtain $s$
	\item $K_{dec}(K_{enc}(s))=s$
	\item Encryption applies a reversible fn to some piece of data, yielding something unreadable
	\item Decryption recovers the original data from the unreadable encryption-output
	\item The encryption/decryption algorithm assumed known; the key is secret
\end{itemize}

\subsection{Substitution Ciphers}
\begin{itemize}
	\item Arbitrary 1:1 mapping of alphabet chars, using a \textbf{substitution table}
	\item All of these are vulnerable to frequency analysis:
	\begin{itemize}
		\item letter
		\item word
		\item common phrases
	\end{itemize}
	\item We could make the substitution table larger (translating $n$-gram, not chars):
	\begin{center}\begin{tabular}{l|l}
		Plaintext & Ciphertext \\
		AAA & QWE \\
		AAB & RTY \\
		AAC & ASD
	\end{tabular}\end{center}
	\item How big is substitution table?
	\begin{itemize}
		\item $A^n$ entries, where A is size of alphabet and $n$ is size of grams
	\end{itemize}
	\item Still vulnerable, but requies more text
\end{itemize}

\subsection{Substitution Rules}
\begin{itemize}
	\item Don't store table explicitly; derive table rows using substitution rule
	\begin{itemize}
		\item e..g, $s \text{XOR} k$, where $k$ is the key
		\item Remember: security level depends on size of key
		\item Key of len $b\geq 2^b$ possible keys
	\end{itemize}
	\item XOR ``flips a bit'' for input bits that correspond to key's ``1''
	\item Encrypted string should ideally show no pattern for frequency analysis attack
	\item What's the right key size?
	\begin{itemize}
		\item Who is trying to break the scheme?
		\item Is that good enough?
	\end{itemize}
\end{itemize}

\subsection{Data Encryption Standard (DES)}
\begin{itemize}
	\item DES is a block cipher with 56-bit key
	\item Data transmitted in 64-bit blocks, each may be coded independently
	\item Triple-DES, breaks up text in 3-56 bit chunks and apply DES to each with different keys
	\img{.7}{sections/lec6/des.png}
\end{itemize}

\subsection{Review of Crypto}
\begin{itemize}
	\item The key is traditional crypto is used to encode the substitution rule
	\begin{itemize}
		\item Needed to encrypt and decrypt
	\end{itemize}
	\item Key distribution is the weak link
	\begin{itemize}
		\item Hard to revoke
		\item Disastrous if ``codebook'' is compromised
		\item Hard to distribute 
		\item Impossible for the Web
	\end{itemize}
\end{itemize}

\subsection{Public-Key Cryptography}
\begin{itemize}
	\item Secure communication without key exchange
	\item Each party has a pair of related keys: \textbf{public} and \textbf{private}
	\begin{itemize}
		\item A public key is published freely
		\item A private key is shared with no one
	\end{itemize}
	\item A message enrypted witho ne can be decrypted with the other
	\item YOu can't derive one from the other (this is critical!)
	\subsubsection{Two Modes}
	\item I encrypt using the public key, and decrypt using the private key
	\img{.8}{sections/lec6/two.png}
	\item Anyone can encrypt; only $S$ can decrypt
	\item Used for data confidentiality
	\item If encrypted using private, anyone with public can decrypt
	\item Used for authenticity
	\item So if someone was able to encrypt, then they must have had private, ensuring no man-in-the-middle
\end{itemize}

\subsection{Trapdoor Functions}
\begin{itemize}
	\item Public key cryptography relies on so-called trapdoor functions
	\begin{itemize}
		\item A fn that is easy to compute, but hard to invert without special info
		\item ``easy'' and ``hard'' meant comutationally
	\end{itemize}
	\item Some poor choices: add, multiply
	\item In practice quite difficult to find good trapdoor rfunctions
	\item Most popular one is related to prime factorizaiton; others possible
	\subsubsection{Prime Factorization}
	\item $n=pq$, where $p$ and $q$ are primes
	\begin{itemize}
		\item Given $p$ and $q$, easy to compute $n$
		\item Given $n$, very hard to find $p$ and $q$
	\end{itemize}
	\subsubsection{Fermat's Little Theorem}
	\item For any prime $p$, and any integer $a$: $a^p = a(\mod p)$
	\subsubsection{Chinese Remainder Theorem}
	\item Consider $x = a_i (\mod p_i), \text{ for }i=1,\ldots, k$
	\item CRT: There's a solution for $x$ if $p_i$ are pairwise relatively prime (i.e., have no common factors greater than 1)
	\item If all $a_i$ are 1, then $x=1(\mod p_i)$
\end{itemize}

\subsection{Cryptography}
\begin{itemize}
	\item We choose two large primes: $p, q$
	\item $n=pq$
	\item Next:
	\begin{itemize}
		\item Set $\lambda = (p-1)(q-1)$
		\item Choose $e$ randomly, such that $e<\lambda$
		\item Choose $d$, such that $de=1(\mod \lambda)$
	\end{itemize}
	\item $n$ and $e$ serve as public key
	\begin{itemize}
		\item $n$ is product of primes $p, q$
	\end{itemize}
	\item $n$ and $d$ serve as private key
	\begin{itemize}
		\item Choosing $d$ requires $e$ and $\lambda$
	\end{itemize}
	\item $\text{encrypt}_{n,e}(m)=m^e (\mod n)=c$
	\item $\text{decrypt}_{n,d}(c)=c^D(\mod n)= m$
	\item Slower than DES because exponential instead of XOR or SHIFTs
	\item Will decryption always work?
	\begin{itemize}
		\item $c^d = m^{de}=m^{k\lambda + 1}$
		\item $m^{k\lambda + 1}=m(m^{(p-1)(q-1)})k$
		\item Recall from Fermat that:
		\begin{itemize}
			\item $m^{p-1}=1(\mod p)$
			\item $m^{q-1}=1(\mod q)$
		\end{itemize}
		\item Which implies:
		\begin{itemize}
			\item $m^{(p-1)(q-1)}=1(\mod p)$
			\item $m^{(p-1)(q-1)}=1(\mod q)$
		\end{itemize}
		\item Use the CRT to combine above 2 equations:
			$$m^{(p-1)(q-1)}=1(\mod n),\text{ where } n=pq$$
		\item $c^d=m(1)^k(\mod n)$
		\item $c^d=m(\mod n)$
		\item Thus, decrypted ciphertext $c=\text{msg}(m)$
	\end{itemize}
	\item Send private shared keys via expensive method, then use cheap method once key is shared. HTTPS does this.
\end{itemize}

\section{The Standard Template Library}
	\subsection{Passing Pointers to Functions}
\begin{itemize}
	\item Pointers can be arguments to functions. For example, suppose you want a function that adds one to an integer argument passed by reference:
\begin{lstlisting}[style=C++]
void add_one(int *x){
	// MODIFIES: *x
	// EFFECTS: adds one to *x
	*x = *x + 1;
}
\end{lstlisting}
	\item If you were to call this function as so: \lstinline[style=C++]{add_one(bar);}, where \lstinline[style=C++]{bar} is a pointer to \lstinline[style=C++]{foo}
\begin{lstlisting}[style=C++]
int foo;
int *bar;
bar = &foo;

add_one(bar);
\end{lstlisting}
	\begin{itemize}
		\item The variable bar is bassed by \textbf{value}, but it's a pointer!
		\item Both bar and the copy of bar refer to the same address in memory.
	\end{itemize}
	\item You can also make the call without the ``middleman'' like: \lstinline[style=C++]{add_one(&foo);}
\end{itemize}

\subsection{Pointer question}
\begin{itemize}
	\item If you modify \lstinline[style=C++]{add_one} to:
\begin{lstlisting}[style=C++]
void add_one(int *x){
	x = x + 1;
}
\end{lstlisting}	
	\item It will increment the value \textbf{of the pointer} by one.
	\item Pointer arithmetic is done based on units of the \textbf{referent type} (the type of the objects in the list).
\end{itemize}

\subsection{Pointers vs. references}
\begin{itemize}
	\item Both allow you to pass objects by reference.
	\item Pointers require some extra syntax at calling time (\&), in the argument list (*), and with each use (*); references only require extra syntax in the argument list (\&).
	\item You can change the object to which a pointer points using arithmetic/assignment, but you cannot change the object to which a reference refers.
	\item You might wonder why you’d ever want to use pointers, since theyrequire extra typing, and allow you to shoot yourself in the foot.
	\item Why use pointers?
	\begin{itemize}
		\item Array variables are internally implemented using pointers
		\item They allow us to create structures (unlike arrays) whose size is not known in advance; we won't see that use until the last third of the course.
	\end{itemize}
\end{itemize}

\subsection{Pointers and Arrays}
\begin{itemize}
	\item Arrays are actually represented via pointers as so:
	\begin{center}
		\includegraphics{sections/lec7/array.png}
	\end{center}
	\item If you were to look at the value of the variable ``array'' (not \lstinline[style=C++]{array[0]}) you'd find that it was exactly the same as the address of \lstinline[style=C++]{array[0]}.
	\item When the argument \lstinline[style=C++]{array} is passed to the function sum, a pointer to the first element of the array is really passed and the compiler does all the work of translating something like: \lstinline[style=C++]{array[3]} into the proper arithmetic/dereference to get the right value.
\begin{lstlisting}[style=C++]
x = array[3];
// Is equivalent to:
int *tmp;
tmp = array + 3;
x = *tmp;
// Or simply:
x = *(array + 3);
\end{lstlisting}
\end{itemize}

\subsection{Indexing vs. pointer arithmetic}
\begin{itemize}
	\item Using array indexing:
\begin{lstlisting}[style=C++]
for (int i = 0; i < SIZE; ++i){
	cout << array[i] << " ";
}
\end{lstlisting}
	\item Using pointer arithmetic:
\begin{lstlisting}[style=C++]
for (int *i = array; i < array+SIZE; ++i){
	cout << *i << " ";
}
\end{lstlisting}
\end{itemize}

\subsection{Array Traversal Using Pointers}
\begin{lstlisting}[style=C++]
int strlen(char *s) {
	char *p = s;
	while (*p) ++p;
	return p - s;
}
\end{lstlisting}
\begin{itemize}
	\item \lstinline[style=C++]{*p} evalues to ``false'' if \lstinline[style=C++]{p} points to a NULL, true otherwise.
	\item \lstinline[style=C++]{++p} advances by ``one character''
	\item \lstinline[style=C++]{p-s} computes the ``number of characters'' between \lstinline[style=C++]{p} and \lstinline[style=C++]{s}
\end{itemize}

\subsection{Constants}
\begin{itemize}
	\item \lstinline[style=C++]{void strcpy(char *dest, const char *src);}
	\item \lstinline[style=C++]{const} is a \textbf{type qualifier} - something that modifies a type
	\item It means ``you cannot change this value once you have initialized it.''
	\item When you have pointers, there are two things you might change:
	\begin{itemize}
		\item The value of the pointer.
		\item The value of the object to which the pointer points.
	\end{itemize}
	\item Either (or both) can be made unchangeable:
\begin{lstlisting}[style=C++]
const T *p;			// "T" (the pointed-to object) cannot be changed
T *const p;			// "p" (the pointer) cannot be changed
const T *const p;	// neither can be changed.
\end{lstlisting}
	\item Adding \lstinline[style=C++]{const} will stop changing value mistakes, and the compiler will catch them.
	\item You can use a pointer-to-T anywhere you expect a pointer-to-const-T, but NOT vice versa
	\item That's because code that expects a pointer-to-T might try to change the T, but this is illegal for a pointer-to-const-T.
	\item However, code that expects a pointer-to-const-T will work perfectly well for a pointer-to-T; it's just guaranteed not to try to change it.
\end{itemize}

\subsection{C strings vs. C++ strings}
\begin{center}
\begin{tabular}[breaklines=true]{p{5cm}|p{5cm}|p{5cm}}
	& C string & C++ string \\
	\hline
	Library headers & 
{\begin{lstlisting}[style=C++] 
#include <string>
\end{lstlisting}} & 
{\begin{lstlisting}[style=C++]
#include string
\end{lstlisting}}\\	
	string constant &
{\begin{lstlisting}[style=C++]
constchar* hello = "hello";
\end{lstlisting}}&
{\begin{lstlisting}[style=C++]
conststring hello = "hello";
\end{lstlisting}} \\
	length &
{\begin{lstlisting}[style=C++]
strlen(hello);//5
\end{lstlisting}} &
{\begin{lstlisting}[style=C++]
hello.length();//5
\end{lstlisting}} \\
	local variable &
{\begin{lstlisting}[style=C++]
constintMAXSIZE=1024; 
char s[MAXSIZE];
\end{lstlisting}} &
{\begin{lstlisting}[style=C++]
string s;
\end{lstlisting}} \\
	copy &
{\begin{lstlisting}[style=C++]
strcpy(s, hello);
\end{lstlisting}} &
{\begin{lstlisting}[style=C++]
s = hello;
\end{lstlisting}} \\
	concatenate &
{\begin{lstlisting}[style=C++]
constchar* world = " world";
char message[MAXSIZE];
strcpy(message, hello);
strcat(message, world);
\end{lstlisting}} &
{\begin{lstlisting}[style=C++]
string message = hello + " world";
\end{lstlisting}} \\
	compare &
{\begin{lstlisting}[style=C++]
if (strcmp(a,b) == 0)
	// do something
\end{lstlisting}} &
{\begin{lstlisting}[style=C++]
if (a == b)
	// do something
\end{lstlisting}} \\
	convert to C++ string &
{\begin{lstlisting}[style=C++]
string cpp_str = hello;
\end{lstlisting}} &
{\begin{lstlisting}[style=C++]
char c_str[MAXSIZE];
strcpy(c_str, message.c_str());
\end{lstlisting}} \\
\end{tabular}
\end{center}

\subsection{Type Sizes}
\begin{itemize}
	\item The amount of memory assigned to a data type is a source of innumerable ``portability bugs'' in programs.
	\item There are \textbf{some} guarantees, however:
	\begin{itemize}
		\item A ``char'' is always one byte
		\item A ``short'' is always at least as big as a char
		\item An ``int'' is always at least as big as a short
		\item A ``long'' is always at least as big as an int
	\end{itemize}
	\item \lstinline[style=C++]{sizeof(int)} tells you the number of bytes required to store an \lstinline[style=C++]{int}
	\begin{center}
		\includegraphics{sections/lec7/type.png}
	\end{center}
\end{itemize}

\end{document}