\subsection{Recap - Bounds Checking for Arrays Example}
\subsubsection{Basic Layout}
\begin{lstlisting}[style=C++]
class Array {
 public:
    Array(unsigned len = 0) : length(len) {
        data = (len ? new double[len] : nullptr);
    }
 private:
    double *data;        // array data
    unsigned int length; // array size
}
\end{lstlisting}
You could use a struct, but users of your class would be able to access all member variables of your struct by default.

\subsubsection{Copy Constructor}
To copy the array, you need to perform a \textit{deep copy}:
\begin{lstlisting}[style=C++]
Array(const Array & a) {
    length = a.getLength(); // getter for length // 1 step
    data = new double[length];                   // 1 step
    for (unsigned i = 0; i < length; i++) {      // n times
        data[i] = a[i];                          // c steps
    }
}
\end{lstlisting}
The complexity of this is: $\O{! + 1 + (nc) + 1}=\O{n}$

\subsubsection{Better Copying}
\begin{lstlisting}[style=C++]
void copyFrom(const Array & a) {
    if (length != a.length) {
        delete[] data;
        length = a.length;
        data = new double[length];
    }

    for (unsigned int i = 0; i < length; i++) {
        data[i] = a.data[i];
    }
}

Array(const Array & a) : length(0), data(nullptr) {
    copyFrom(a);
}

Array & operator=(const Array & a) {
    copyFrom(a);
    return *this;
}
\end{lstlisting}

\subsection{The Big 5}
\begin{itemize}
	\item Destructor
	\item Copy constructor
	\item Overloaded operator=
	\item Copy constructor from \textit{r-value}
	\item Overloaded operator= from \textit{r-value}
\end{itemize}

\subsubsection{Destructor}
\begin{lstlisting}[style=C++]
~Array() {
    if (data != nullptr) {
        for (int i = 0; i < length; i++) { // n times
            delete data[i];                // 1 step
        }
        delete[] date;                     // 1 step
        data = nullptr;                    // 1 step
    }
}
\end{lstlisting}
Complexity: $\O{n}$

\subsubsection{Access Operator[]}
\begin{lstlisting}[style=C++]
const double & operator[](int idx) const {
    if (idx < length && idx >= 0)
        return data[idx];
    throw runtime_error("bad idx");
}
\end{lstlisting}
\begin{itemize}
	\item Declares read-only access/
	\item Automatically selected by the compiler when an array being access is marked \lstinline[style=C++]{const}
	\item Helps compiler optimize code for speed
	\item Some functions, like \lstinline[style=C++]{ostream &operator<<(ostream &os, const Array &a)} require \lstinline[style=C++]{const}.
	\item Must also make a non-constr version
\end{itemize}

\subsubsection{Access with more than 2 dimensions}
\begin{lstlisting}[style=C++]
const double &operator()(int i, int j) const {
    // return by const reference
}
\end{lstlisting}
\begin{itemize}
	\item Can't overload \lstinline[style=C++]{operator[][]}
\end{itemize}

\subsubsection{Inserting an Element}
\begin{lstlisting}[style=C++]
bool insert(int index, double val) {
    if (index >= size || index < 0) {
        return false;
    } else {
        for (int i = size - 1; i > index; --i) {
            data[i] = data[i - 1];
        }
        data[index] = val;
        return true;
    }
}
\end{lstlisting}
Complexity:
\begin{itemize}
	\item Best case: $\O{1}$
	\begin{itemize}
		\item The element is inserted at the end
	\end{itemize}

	\item Average case: $\O{n}$
	\begin{itemize}
		\item Go through half of elements (but that is a constant times $n$)
	\end{itemize}

	\item Worst case: $\O{n}$
	\begin{itemize}
		\item Go through all elements
	\end{itemize}
\end{itemize}

\subsubsection{Append}
When array is full, resize:
\begin{itemize}
	\item Double array size from $n$ to $2n$
	\item Copy $n$ items from the original array to the new array
	\item Appending $n$ elements
	\item Total: $1 + n + n = 2n + 1$ steps
	\item Amortized: $(2n+1)/n=\O{1}$ steps
\end{itemize}

\subsection{Linked Lists and Iterators}
\begin{itemize}
	\item \textbf{Linked list}: Each person points to the next person
	\item \textbf{Doubly linked list}: Each person points to each other
	\item \textbf{Circularly linked list}: A list which has the first and last nodes pointing to each other
\end{itemize}

\subsubsection{Arrays vs Linked Lists}
Arrays:
\begin{itemize}
	\item Access:
	\begin{itemize}
		\item Random: $\O{1}$ time
		\item Sequential: $\O{1}$ time
	\end{itemize}

	\item Insert:
	\begin{itemize}
		\item Insert: $\O{n}$ time
		\item Append: $\O{n}$ time
	\end{itemize}

	\item Book-keeping:
	\begin{itemize}
		\item \lstinline[style=C++]{ptr} to beginning
		\item \lstinline[style=C++]{current_size} or \lstinline[style=C++]{ptr}
		\item \lstinline[style=C++]{max_size} or \lstinline[style=C++]{ptr} to end off allocated space
	\end{itemize}

	\item Memory:
	\begin{itemize}
		\item Wastes memory if size is too large
		\item Requires reallocation if too small
	\end{itemize}
\end{itemize}
Linked Lists:
\begin{itemize}
	\item Access:
	\begin{itemize}
		\item Random: $\O{n}$ time
		\item Sequential: $\O{1}$ time
	\end{itemize}

	\item Insert:
	\begin{itemize}
		\item Insert: $\O{n}$ time
		\item Append: $\O{1}$ time
		\begin{itemize}
			\item Append to \lstinline[style=C++]{tail_ptr}
		\end{itemize}
	\end{itemize}

	\item Book-keeping:
	\begin{itemize}
		\item \lstinline[style=C++]{head_ptr} to first node
		\item \lstinline[style=C++]{size} (optional)
		\item \lstinline[style=C++]{tail_ptr} to last node
		\item In each node, \lstinline[style=C++]{ptr} to next node
		\item Wasteful for small data times (overhead on pointers)
	\end{itemize}

	\item Memory:
	\begin{itemize}
		\item Allocates memory as needed
		\item Requires memory for pointers
	\end{itemize}
\end{itemize}

\subsection{Linked List Implementation}
\begin{lstlisting}[style=C++]
class LinkedList {
 public:
    LinkedList();
    ~LinkedList();
 private:
    struct Node {
        double item;
        Node *next;
        Node() {next = nullptr;}
    };

    Node *head_ptr;
};
\end{lstlisting}

\subsubsection{Methods}
\begin{lstlisting}[style=C++]
int size() const;

bool append_item(double item);
bool append_node(Node *n);

bool delete_item(double item);
bool delete_node(Node *n);
\end{lstlisting}

\subsubsection{Maintaining Consistency}
Arrays:
\begin{itemize}
	\item Stored size matches number of elements at all times
	\item Be sure that \lstinline[style=C++]{start_ptr + size < end_ptr}
\end{itemize}
Linked Lists
\begin{itemize}
	\item Stored size matches number of elements
	\item The last node points to \lstinline[style=C++]{nullptr}
	\item If the list is a circular list, last node points to head
	\item If a doubly-linked list, next/prev pointers are consistent
\end{itemize}