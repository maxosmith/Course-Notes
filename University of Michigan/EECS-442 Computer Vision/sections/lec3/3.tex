\subsection{Pinhole Camera}
\img{.75}{sections/lec3/ap.png}
\begin{itemize}
	\item A barrier is added to block off most of the rays
	\item This process reduces blurring resulting from multiple rays of light hitting the same spot
	\item \textbf{Aperature}: hole in the barrier that allows light to pass through.
\img{.75}{sections/lec3/pin.png}
	\item $f$ (\textbf{focal length}): distance between the lens (pinhole) the image sensor (image plane)
	\item $c$: center of the camera/pinhole/aperature
	\item Say we have an object of interest $P$ and want to understand how it'll register on our image plane as $P'$
	$$P=\begin{bmatrix}
		x \\ y \\ z
	\end{bmatrix}\to P'=\begin{bmatrix}
		x' \\ y'
	\end{bmatrix}$$
\img{.75}{sections/lec3/obj.png}
	\item We can derive the tranformation using similar triangles:
	$$x'=f\frac{x}{z}, y'=f\frac{y}{z}$$
\end{itemize}

\subsection{Cameras \& Lenses}
\begin{itemize}
	\item If the aperature is too large, your image will blur
	\item If it's too small you won't have enough light going through to have an image (too dark)
	\item This can be fixed by using lenses 
\img{.75}{sections/lec3/lens.png}
	\item A \textbf{lens} focuses light onto the film
\img{.75}{sections/lec3/par.png}
	\item Rays passing through the center of the lens are not deviated
	\item All parallel rays converge to one point on a plan located at the \textbf{focal length} $f$
\img{.75}{sections/lec3/c.png}
	\item There is a specific distance at which objects are ``in focus''
	\item Related to the concept of depth of field (``range object is in focus'')
	\item The area pointed at in the image refers to the ``circle of confusion'' an area where you can't resolve individual rays
\end{itemize}

\subsection{Snell's Law}
\begin{itemize}
	\item Light bends at an interface as a result of travelling different speeds in different mediums.
\img{.75}{sections/lec3/snell.png}
	$$n_1 \sin\alpha_1=n_2\sin \alpha_2$$
	\item $\alpha_1$: incident angle
	\item $\alpha_2$: refraction angle
	\item $n_i$: index of refraction
	\item For small angles:
	\img{1}{sections/lec3/thin.png}
	\begin{itemize}
		\item $n_1\alpha_1\approx n_2\alpha_2$
		\item $z' = f+z_0$
		\item $f = \frac{R}{2(n-1)}$
		$$x' = z'\frac{x}{z}, y'=z'\frac{y}{z}$$
	\end{itemize}
\end{itemize}

\subsection{Issues with lenses}
\begin{itemize}
	\item \textbf{Chromatic aberration}: lenses have different refractive indices for different wavelengths which cause color fringing
	\img{.8}{sections/lec3/fring.png}
	\item \textbf{Spherical aberration}: rays farther from the optical axis focus closer 
	\img{1}{sections/lec3/sph.png}
	\item \textbf{Radial distortion}: deviations are most noticeable for rays that pass through the edge of the lens
	\img{1}{sections/lec3/rad.png}
\end{itemize}

\subsection{Converting to pixels}
\img{.8}{sections/lec3/pin.png}
\img{.8}{sections/lec3/coor.png}
\begin{enumerate}
	\item Offset
	\begin{itemize}
		\item The center of the image plane is at coordinates $C=(c_x, c_y)$
		\item This modifies our projection mapping to: $(f\frac{x}{z}+c_x, f\frac{y}{z}+c_y)$
	\end{itemize}
	\item From metric to pixels
	\begin{itemize}
		\item If we let $k, l$ be conversion metrics with units pixel/m, too allow for pixel length and width to be different
		\item We can now make these modifications to the conversion to be in pixels:
		$$(x,y,z)\to(fk\frac{x}{z}+c_x, fl\frac{y}{z}+c_y)\\
		(x,y,z)\to(\alpha \frac{x}{z}+c_x, \beta\frac{y}{z}+c_y)$$
		\item We use $\alpha, \beta$ for ease of notation 
	\end{itemize}
\end{enumerate}

\subsection{Linear Transformation to pixels}
\begin{itemize}
	\item Our previous formula isn't linear because of the division by $z$
	\item If we use homogeneous coordinates we can make it linear:
	\begin{itemize}
		\item Image coordinates: $(x, y, 1)$
		\item Scene coordinates: $(x, y, z, 1)$
	\end{itemize}
	\item We can now define a matrix $M$ to represent our transformation:
	$$x'=Mx\\
	x'=\begin{bmatrix}
		fx \\ fy \\z
	\end{bmatrix}=\begin{bmatrix}
		f & 0 & 0 & 0 \\
		0 & f & 0 & 0 \\
		0 & 0 & 1 & 0
	\end{bmatrix}\begin{bmatrix}
		x \\ y \\ z \\ 1
	\end{bmatrix}$$
	\item We can now use this idea for our pixel mapping:
	$$x'=\begin{bmatrix}
		\alpha x+c_x z \\ \beta y + c_y z \\ z
	\end{bmatrix} = \begin{bmatrix}
		\alpha & 0 & c_x & 0\\
		0 & \beta & c_y & 0 \\
		0 & 0 & 1 & 0
	\end{bmatrix}\begin{bmatrix}
		x \\ y \\ z \\ 1
	\end{bmatrix}$$
\end{itemize}

\subsection{Camera Matrix}
\begin{itemize}
	\item The subsection of our previous matrix:
	$$\begin{bmatrix}
		\alpha & 0 & c_x \\
		0 & \beta & c_y \\
		0 & 0 & 1 
	\end{bmatrix}=K$$
	\item $K$ is called the camera matrix
	\item It sometimes features a skew parameter ($s$), but in pratcie $s\approx 0$
		$$\begin{bmatrix}
		\alpha & s & c_x \\
		0 & \beta & c_y \\
		0 & 0 & 1 
	\end{bmatrix}$$
	\img{.7}{sections/lec3/skew.png}
\end{itemize}

\subsection{3D Rotation of Points}
\begin{itemize}
	\item Rotation around the coordinate axes, \textbf{counter-clockwise}:
	$$R_x(\alpha)=\begin{bmatrix}
		1 & 0 & 0 \\
		0 & \cos(\alpha) & -\sin(\alpha) \\
		0 & \sin(\alpha) & \cos(\alpha)
	\end{bmatrix}\\
	R_y(\beta)=\begin{bmatrix}
		\cos(\beta) & 0 & \sin(\beta) \\
		0 & 1 & 0 \\
		-\sin(\beta) & 0 & \cos(\beta)
	\end{bmatrix}\\
	R_z(\gamma)=\begin{bmatrix}
		\cos(\gamma) & -\sin(\gamma) & 0 \\
		\sin(\gamma) & \cos(\gamma) & 0 \\
		0 & 0 & 1
	\end{bmatrix}$$
\end{itemize}

\subsection{Camera Matrix Homogeneous Coordinates}
\begin{itemize}
	\item In 4D homogeneous coordinates the mapping from world to image follows:
	$$X=\begin{bmatrix}
		R & T \\
		0 & T
	\end{bmatrix}_{4\times 4} X_{world}$$
	$$X'=K_{3\times 3}[R\text{ }T]_{3\times 4} X_{world, 4\times 1}$$
	\item Here $K$ represents the internal camera parameters: $K=\begin{bmatrix}
		\alpha & s & c_x \\
		0 & \beta & c_y \\
		0 & 0 & 1
	\end{bmatrix}$
	\item Our external camera parameters are represented by $R, T$
	\item If we represent $M=\begin{bmatrix}
		\vec{m}_1 & \vec{m}_2 & \vec{m}_3
	\end{bmatrix}^{T}$, then we may also write our transformation as:
	$$(x,y,z)_w\to (\frac{\vec{m}_1 X_m}{\vec{m}_3 X_w}, \frac{\vec{m}_2 X_w}{\vec{m}_3 X_w} )$$
\end{itemize}

\begin{theorem}[Theorem (Faaugeras, 1993)]
	$$M=K[R\text{ }T]=[KR\text{ }KT]=[A\text{ }b]$$
\begin{itemize}
	\item Perspective projection req:
	\begin{itemize}
		\item $det(A)\neq 0$
	\end{itemize}
	\item Zero-skew conditions:
	\begin{itemize}
		\item $det(A)\neq 0$
		\item $(\vec{a}_1\times \vec{a}_3)\cdot(\vec{a}_2\times \vec{a}_3)=0$
	\end{itemize}
	\item Perspective proj \& zero-skew conditons:
	\begin{itemize}
		\item $det(A)\neq 0$
		\item $(\vec{a}_1\times \vec{a}_3)\cdot(\vec{a}_2\times \vec{a}_3)=0$
		\item $(\vec{a}_1\times \vec{a}_3)\cdot(\vec{a}_1\times \vec{a}_3)=(\vec{a}_2\times \vec{a}_3)\cdot(\vec{a}_2\times \vec{a}_3)$
	\end{itemize}
\end{itemize}
\end{theorem}

\subsection{Properties of Projection}
\begin{itemize}
	\item Points project to points
	\item Liens project to lines
	\item Distant objects look smaller
	\img{1}{sections/lec3/small.png}
	\item Angles are not preserved
	\item Parallel lines will meet at a \textbf{vanishing point}
	\img{1}{sections/lec3/vanish.png}
	\item Objects on the periphery are expanded
	\item Degenerate cases:
	\begin{itemize}
		\item Line through focal point projects to a point
		\item Plane though focal point projects to line
		\item Plane perpendicular to image plane projects to part of the image (with horizon)
	\end{itemize}
\subsubsection{Vanishing points}
	\item Sets of parallel lines on the same plan lead to collinear vanishing points (The line is called the \textbf{horizon} for that plane)
	\item Each set of parallel lines meets at a different point (The vanishing point for this direction)
\end{itemize}

\subsection{Other camera models}
\begin{itemize}
\subsubsection{Weak perspective projection}
	\item When the relative scene depth is small compared to its distance from the camera
	\item Namely, if the relative distance (scene depth) between two points of a 3D object along the optical axis is much smaller than the average distance to the camera
	\img{1}{sections/lec3/weak.png}
	$$x'=-\frac{f'}{z_0}x, y'=-\frac{f'}{z_0}y$$
	\item In this perspective the fraction is the \textbf{magnification} ($m$)
	\item Accurate when object is small and distant
	\item Most useful for recognition
\subsubsection{Orthographic (affine) projection}
	\item Distance form center of projection to image plane is infinite
	\img{.7}{sections/lec3/orth.png}
	$$x'=-x, y'=-y$$
	\item All rays are parallel
\end{itemize}