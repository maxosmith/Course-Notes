\documentclass[11pt,letterpaper]{article}
\usepackage{notes}

%Global Course Variables
\newcommand{\myCourseCode}{EECS 381}
\newcommand{\myCourseName}{Objecty-Oriented and Advanced Programming}
\newcommand{\myProf}{David Kieras (Maybe)}
\newcommand{\myTerm}{Fall 2015 (Maybe)}

%Headers
\lhead{\myCourseName}
\rhead{\fancyplain{}{\rightmark}} 

%Footers
\cfoot{\thepage}

\begin{document}
\titleHeader{\myCourseCode}{\myCourseName}{\myProf}{\myTerm}

%Document information
\rule[0.5ex]{1\columnwidth}{.5pt}
Contributors: Max Smith
\begin{center}
	Latest revision: \today
\end{center}
\toc
\abstr{Check}

%----------------------------
%Document Begins
%----------------------------

\section{Reading 1}
Unnamed Examples:

\begin{defn}
	A relative path is a path that is referring to your current directory.
\end{defn}

\begin{rem}
	Recall that double quotes do not protect back quotes.
\end{rem}

\begin{exmp}
	$$2+2=3$$
\end{exmp}

With some names:
\begin{defn}[Relative Path]
    A relative path is a path that is referring to your current directory.
\end{defn}

\begin{rem}[Double Quotes]
    Recall that double quotes do not protect back quotes.
\end{rem}

\begin{exmp}[Addition]
    $$2+2=3$$
\end{exmp}

\subsection{Chapter 1}

\subsection{Chapter 2}
Restrictions on variable names:\newline
Internal variables have a can have a length of 31 characters and can only be comprised of integers and letters and numbers (The underscore counts as a letter). You shouldn't start off variables with an underscore because library routines often use those names. It is traditional practice to use lower case for variable names and all uppercase for symbolic constants. External names can only have a length of 6 characters and one case because they may be used by assemblers and loaders.
Data type lengths:
char - a single byte, 8 bits
int - an integer, size depends on system
short - an integer that is no larger than an integer (typically 16 bits)
long - an integer, size is no smaller tahn an int (typically graeter than or equal to 32 bits)
float - single-precision floating point
double - double precision floating point
unsigned (data type) - respective data type that can only be greater than or equal to 0. 


\newpage

\section{qwertqw}

\section{AGAIN?!?!!}

%----------------------------
%Quick Reference Examples
%----------------------------

\end{document}
